\input{CM_class.tex}

\begin{document}

\begin{frame}
   \titlepage
\end{frame}

\input{CM_introduction.tex}

\begin{frame}
   \tableofcontents
\end{frame}

% =============================================================================
% =============================================================================
\section{Data Analysis}
% 3 Hours course (for Analysis and Pre-processing)
% =============================================================================
% =============================================================================


%------------------------------------------------------------------------------
\subsection{Measures}
%------------------------------------------------------------------------------


\begin{frame}\frametitle{Measures}
   \begin{itemize}
      \item Centrality:
      \begin{itemize}
         \item Goal: representation of the majority's value
         \item Mean (average): average age, mean size
         \item Median: median salary, median patrimony
      \end{itemize}
      \item Dispersion:
      \begin{itemize}
         \item Goal: majority's spread (variation) around the central value
         \item Standard deviation (sqrt variance): financial markets volatility
         \item Interquartile Range (IQR)
         \item Min-Max: job proposal salary
      \end{itemize}
   \end{itemize}
\end{frame}



\subsection{Centrality}
%------------------------------------------------------------------------------

\begin{frame}\frametitle{Pandas - Mean}
   \begin{minipage}{0.58\linewidth}
      \begin{itemize}
         \item Sensible to extreme values
         \begin{itemize}
            \item Age: good representation
            \item Patrimony: biased, not representative
         \end{itemize}
      \end{itemize}
      \vspace{.5cm}
      $$mean = \frac{x_1 + x_2 + ... + x_n}{n}$$
      \begin{figure}[H]
         \includegraphics[width=4cm]{../images/illustrations/data_analysis_df_1.png}
      \end{figure}
   \end{minipage}
   \begin{minipage}{0.38\linewidth}
      \begin{figure}[H]
         \includegraphics[width=5cm]{../images/illustrations/mean.png}
      \end{figure}
   \end{minipage}
\end{frame}


\begin{frame}\frametitle{Pandas - Median}
   \begin{minipage}{0.58\linewidth}
      \begin{itemize}
         \item Insensitive to extreme values
         \begin{itemize}
            \item Age: good representation
            \item Patrimony: good representation of the majority
         \end{itemize}
      \end{itemize}
      \vspace{.5cm}
      Order values, then:
      $$median = \frac{x_{center_2} - x_{center_1}}{2}$$
      \begin{figure}[H]
         \includegraphics[width=4cm]{../images/illustrations/data_analysis_df_1.png}
      \end{figure}
   \end{minipage}
   \begin{minipage}{0.38\linewidth}
      \begin{figure}[H]
         \includegraphics[width=5cm]{../images/illustrations/median.png}
      \end{figure}
   \end{minipage}
\end{frame}



\subsubsection{Dispersion}
%------------------------------------------------------------------------------


\begin{frame}\frametitle{Pandas - Standard Deviation (std)}
   \begin{minipage}{0.58\linewidth}
      \begin{itemize}
         \item Sensible to extreme values
         \item In the unit of the varaible
         \item Interpretable
         \begin{itemize}
            \item Age: good representation
            \item Patrimony: Patrimony: biased, not representative
         \end{itemize}
      \end{itemize}
      \vspace{.5cm}
      $$\overline{x} = mean(values)$$
      $$std = \sqrt{\frac{(x_1-\overline{x}) + ... + (x_n-\overline{x})}{n}}$$
      \begin{figure}[H]
         \includegraphics[width=4cm]{../images/illustrations/data_analysis_df_1.png}
      \end{figure}
   \end{minipage}
   \begin{minipage}{0.38\linewidth}
      \begin{figure}[H]
         \includegraphics[width=5cm]{../images/illustrations/median.png}
      \end{figure}
   \end{minipage}
\end{frame}


\begin{frame}\frametitle{Pandas - Interquartile Range (iqr)}
   \begin{minipage}{0.58\linewidth}
      \begin{itemize}
         \item Insensitive to extreme values
         \item In the unit of the varaible
         \item Interpretable
         \begin{itemize}
            \item Age: good representation
            \item Patrimony: quite good representation
         \end{itemize}
      \end{itemize}
      \vspace{.5cm}
      $$iqr = 3rd\_quantile - 1st\_quantile$$
      \begin{figure}[H]
         \includegraphics[width=3cm]{../images/illustrations/data_analysis_df_2.png}
      \end{figure}
   \end{minipage}
   \begin{minipage}{0.38\linewidth}
      \begin{figure}[H]
         \includegraphics[width=5cm]{../images/illustrations/iqr.png}
      \end{figure}
   \end{minipage}
   \footnote{Quantile values 1st: 25\%, 2nd: 50\%, 3rd: 75\% - after ordering}
\end{frame}



\begin{frame}\frametitle{Pandas - Min Max}
   \begin{minipage}{0.58\linewidth}
      \begin{itemize}
         \item Sensible to extreme values
         \item In the unit of the variable
         \item Interpretable
         \item Easy to compute
         \item Idea of max range
      \end{itemize}
      \vspace{.5cm}
      $$min-max = max(values) - min(values)$$
      \begin{figure}[H]
         \includegraphics[width=3cm]{../images/illustrations/data_analysis_df_2.png}
      \end{figure}
   \end{minipage}
   \begin{minipage}{0.38\linewidth}
      \begin{figure}[H]
         \includegraphics[width=5cm]{../images/illustrations/min_max.png}
      \end{figure}
   \end{minipage}
\end{frame}



%------------------------------------------------------------------------------
\subsection{Patterns Analysis}
%------------------------------------------------------------------------------

\begin{frame}\frametitle{Patterns Analysis}
   \begin{itemize}
      \item Univariate Analysis
      \begin{itemize}
         \item Time Series
         \begin{itemize}
            \item Trend
            \item Seasonality
            \item Auto-correlation
         \end{itemize}
         \item Other quantitative variables
         \item Qualitative variables
      \end{itemize}
      \item Multivariate Analysis (between variables)
      \begin{itemize}
         \item Quantitative variables
         \begin{itemize}
            \item Linear
            \item Non-Linear
         \end{itemize}
         \item Qualitative variables
         %TODO: with ordering, with cardinality
      \end{itemize}
   \end{itemize}
\end{frame}

\begin{frame}\frametitle{Correlation}
   \begin{itemize}
      \item Move repetitively in conjunction
      \item Methods
      \begin{itemize}
         \item Pearson
         \item Spearman (Rank)
      \end{itemize}
      \item Spurious correlation (ice cream, Eiffel Tower)
   \end{itemize}
\end{frame}


%------------------------------------------------------------------------------
\subsection{Statistical Tools}
%------------------------------------------------------------------------------

\subsubsection{Statistical Laws}

\begin{frame}\frametitle{Statistical Laws}
   \begin{itemize}
      \item Normal Law / Gauss Curve
      \begin{itemize}
         \item Totally resumed by mean and variance
         \item Constant mean (0 if centered) and variance (1 if reduced)
         \item Uncorrelated individuals
         \item Symetric (Skewness=0)
         \item Precise bell shape (Kurtosis=3)
      \end{itemize}

      \item Power Laws: multiplicative growth
      % TODO: insert graph examples of both distributions
      \item Examples:
      \begin{itemize}
         \item Normal: human age, size, weight, grades
         \item Power: lakes size, wealth
      \end{itemize}
   \end{itemize}
\end{frame}


\subsubsection{Statistical Tests}

\begin{frame}\frametitle{Statistical Tests}
   \begin{itemize}
      \item Intention: prevent sampling error
      \item Hypothesis (Normal Law)
      \item Examples:
      \begin{itemize}
         \item Normality test
         \item ANOVA
         \item Pearson's r
         \item Chi square
      \end{itemize}
         % TODO: review https://www.scribbr.com/statistics/statistical-tests/
   \end{itemize}
\end{frame}



% =============================================================================
% =============================================================================
\section{Data Pre-Processing}
% 3 Hours course (for Analysis and Pre-processing)
% =============================================================================
% =============================================================================


%------------------------------------------------------------------------------
\subsection{Data Management}
%------------------------------------------------------------------------------

\begin{frame}\frametitle{Data Management}
   \begin{itemize}
      \item Select variables
      \item Merge tables
      % TODO: add multiple code examples
   \end{itemize}
\end{frame}


%------------------------------------------------------------------------------
\subsection{Data Cleaning}
%------------------------------------------------------------------------------

\begin{frame}\frametitle{Data Cleaning}
   \begin{itemize}
      \item NaN Imputation
      \item Outliers
      % TODO: add multiple code examples
   \end{itemize}
\end{frame}


%------------------------------------------------------------------------------
\subsection{Feature Engineering}
%------------------------------------------------------------------------------

\begin{frame}\frametitle{Feature Engineering}
   \begin{itemize}
      \item Quantitative variables (numbers representing quantities): create groups
      \item Qualitative variables (categories): one-hot encode
      \item Filter
      % TODO: add multiple examples
   \end{itemize}
\end{frame}


\end{document}