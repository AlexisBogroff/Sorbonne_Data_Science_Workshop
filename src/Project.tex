\input{TD_class.tex}
\setcounter{section}{0}

\begin{document}
%----------------------------------------------------------------------------------------
%========================================================================================
\section{Project}
%========================================================================================
%----------------------------------------------------------------------------------------


%****************************************************************************************
\subsection{Set up}
%****************************************************************************************

\begin{enumerate}
    \item Load data:\\
    \textit{masse-salariale-et-assiette-chomage-partiel-mensuelles-du-secteur-prive\_modif.csv}\footnote{\href{https://www.data.gouv.fr/fr/datasets/masse-salariale-et-assiette-chomage-partiel-mensuelles-du-secteur-prive/}{Data is a modified version from this source}}
    \item See number of samples (rows) and features (columns)
    \item See data type
    \item Set \textit{dernier\_jour\_du\_mois} as index
    \item Cast index as datetime
    \item Sort index in ascending order
\end{enumerate}


%****************************************************************************************
\subsection{Data Analysis}
%****************************************************************************************

\begin{enumerate}
    \item Discover data:
    \begin{itemize}
        \item Visualize (plot) data (can be done in one simple line of code)
    \end{itemize}
\end{enumerate}

%****************************************************************************************
\subsection{Data Cleaning}
%****************************************************************************************

\begin{enumerate}
    \item Check for missing values (one might be more subtle than a yelling NaN)
    \item Impute these missing values with at least 2 methods seen in the lectures, 
    don't delete them in this project (imputing is more difficult than deleting)
    \item Check and treat outlier(s)
\end{enumerate}

%****************************************************************************************
\subsection{Feature Engineering}
%****************************************************************************************

\begin{enumerate}
    \item Add a feature \textit{is\_year\_end}
    \begin{itemize}
        \item 1 when month is november or december
        \item 0 otherwise
    \end{itemize}
\end{enumerate}

%****************************************************************************************
\subsection{Prediction}
%****************************************************************************************

\begin{enumerate}
    \item Split your data into a train set (70\% of data) and a test set (30\%)
    \item Use a linear regression to predcit \textit{part\_assiette\_chomage\_partiel} 1 month ahead
    \begin{itemize}
        \item you should shift your features (in time) compared to your target
        \item find tutorials, there are a lot of them, its the only way toward autonomous learning!
    \end{itemize}
    \item How good is your prediction?
    \begin{itemize}
        \item Use metric(s) to evaluate your model on both the train and test sets
        \item Interpret the results
        \item Give advices to your (hypothetical) colleague to continue your work
    \end{itemize}
\end{enumerate}



%----------------------------------------------------------------------------------------
\subsubsection{Bonus}
%----------------------------------------------------------------------------------------

\begin{enumerate}
    \item Make a prediction without the added variable \textit{is\_year\_end}
    \item Use a Ridge regression in place of the Linear regression (you might become happy about the results!)
    \item Use a \textbf{polynomial} regression to predcit 1 month ahead (find tutorials, there are a lot of them, and its the only way to learn autonomously!)
    \item Predict 2 months ahead, then 3 and 4 months ahead. If your code is written correctly, it should only require to manually change the value of a constant.
\end{enumerate}


\end{document}