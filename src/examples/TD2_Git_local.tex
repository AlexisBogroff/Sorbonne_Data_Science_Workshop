\documentclass{article}
\usepackage{ae,lmodern}
\usepackage[english]{babel}
\usepackage[utf8]{inputenc}
\usepackage[T1]{fontenc}

\usepackage{caption}
\captionsetup[figure]{labelformat=empty}

\PassOptionsToPackage{usenames,dvipsnames}{xcolor}
\usepackage{xcolor,colortbl}
\definecolor{DarkGrey}{HTML}{222222}
\definecolor{DarkBlue}{HTML}{004BA9}
\definecolor{DarkRed}{HTML}{CC1111}
\definecolor{DarkGreen}{HTML}{117711}
\definecolor{DarkOrange}{HTML}{CC7000}
\definecolor{LightGrey}{HTML}{DDDDDD}
\definecolor{LightBlue}{HTML}{F0F8FF}
\definecolor{codegreen}{rgb}{0,0.6,0}
\definecolor{codepurple}{rgb}{0.58,0,0.82}

\usepackage[cache=false]{minted}
\setminted[bash]{
   bgcolor=LightBlue,
   breaklines, breakanywhere,
   frame=single,
   autogobble
}
\usemintedstyle[python]{native}
\setminted[python]{
   bgcolor=black,
   breaklines, breakanywhere,
   autogobble
}

\usepackage{listings}
\usepackage{lstautogobble}
\lstdefinestyle{bash}{
    backgroundcolor=\color{DarkGrey},   
    commentstyle=\color{codegreen},
    keywordstyle=\color{magenta},
    numberstyle=\tiny\color{DarkGrey},
    stringstyle=\color{codepurple},
    basicstyle=\ttfamily\tiny\color{LightGrey},
    escapeinside={\%*}{*)},
    breakatwhitespace=false,         
    breaklines=true,                 
    captionpos=b,                    
    keepspaces=true,                 
    numbers=left,                    
    numbersep=5pt,                  
    showspaces=false,                
    showstringspaces=false,
    showtabs=false,
    showlines=false,
    tabsize=2
}

\usepackage{tikz}
\usetikzlibrary{calc,decorations.pathreplacing,arrows,arrows.meta,shapes,patterns, positioning}
\newcommand\BigLength{14.6em}
\newcommand\Height{2em}
\newcommand\Sep{0.6em}
\newcommand\Center{\BigLength*1/2}
\newcommand\BigBox{\BigLength+\Sep}
\newcommand\HalfBox{\BigLength*1/2-\Sep*1/4}
\newcommand\HalfLength{\BigLength*1/2-\Sep*5/4}
\newcommand\CenterL{\BigLength*1/4-\Sep*1/8}
\newcommand\CenterR{\BigLength*3/4+\Sep*1/8}
\tikzstyle{layer}=[rectangle,thick,text centered,
                     minimum height=\Height,minimum width=\BigLength]
\tikzstyle{short}=[rectangle,thick,text centered,
                     minimum height=\Height,minimum width=\HalfLength]
\tikzstyle{dibox}=[rectangle,thick,semitransparent,
                     minimum height=(\Height+\Sep)*2,minimum width=\BigBox]
\tikzstyle{vmbox}=[rectangle,thick,semitransparent,
                     minimum height=(\Height+\Sep)*3,minimum width=\HalfBox]
\tikzstyle{ctbox}=[rectangle,thick,semitransparent,
                     minimum height=(\Height+\Sep)*2,minimum width=\HalfBox]
\tikzstyle{vebox}=[rectangle,thick,semitransparent,
                     minimum height=(\Height+\Sep)*1,minimum width=\HalfBox]

\usepackage{hyperref}
\usepackage{grffile}

\renewcommand{\thesection}{TD\arabic{section}/5:}
\renewcommand{\thesubsection}{Exercise \arabic{subsection}:}


\setcounter{section}{1}

\begin{document}

%----------------------------------------------------------------------------------------
\section{Git Filesystem \& Commits}
%----------------------------------------------------------------------------------------

%----------------------------------------------------------------------------------------
\subsection{Configure Git}  % 5 to 10 min
%----------------------------------------------------------------------------------------
Using only command-line in your Linux shell,
\begin{enumerate}
    \item Check that Git is installed on your environment.
    \item Configure your name and e-mail globally.
    \item Check that Git has correctly recorded these two pieces of information. \\
    \textit{hint: All Git commands have a \textbf{-h} flag to display the corresponding help.
    Look there for the option of the \textbf{git config} command that lists all Git configuration.}
\end{enumerate}

\ifdefined\answer
\begin{minted}{bash}
    git
    git config --global user.name "Bernard Lerenard"
    git config --global user.email "bernard@yahoo.fr"
    git config -h
    git config --list
\end{minted}
\fi

%----------------------------------------------------------------------------------------
\subsection{Hashing}  % 30 to 45 min
%----------------------------------------------------------------------------------------
Using only command-line in your Linux shell and without creating a Git repository for the moment,
\begin{enumerate}
    \item Create a root directory.
    \item Create a text file inside whose content is ``Hello World''.
    \item What is the size of the file ?
    \item Display the file content on the screen.
    \item Compute the SHA-1 hash of the file content. \\
        \textit{hint: You can use the GNU core utilities \textbf{sha1sum} command.}
    \item What hash would Git compute on this file ? \\
        \textit{hint: You can use the \textbf{git hash-object} plumbing command
        (no need to create a Git repository for the moment).} \\
        \\
        They are different aren't they ? \\
        Actually Git prepends 2 properties to the file content before hashing, compressing and saving it:
        \begin{enumerate}
            \item the Git object type followed by a space character
            \item the file size followed by a null character (\textbackslash0)
        \end{enumerate}
    \item Create a second file whose content is what Git would really consider when saving your first file. \\
        \textit{hint: The \textbf{echo} command has a \textbf{-e} flag to interpret backslash escapes.}
    \item Compute the SHA-1 hash of this second file and check it is equal to the Git hash of your first file.
\end{enumerate}

\ifdefined\answer
\begin{minted}{bash}
    mkdir exercice_root
    cd exercice_root/
    echo "Hello World" > file1.txt
    ls -l                                             -> 12
    cat file1.txt
    sha1sum file1.txt                                 -> 648a6a6ffffdaa0badb23b8baf90b6168dd16b3a
    git hash-object file1.txt                         -> 557db03de997c86a4a028e1ebd3a1ceb225be238
    echo -e "blob 12\0$(<file1.txt)" > file1_git
    sha1sum file1_git | awk '{print $1}'              -> 557db03de997c86a4a028e1ebd3a1ceb225be238
\end{minted}
\fi

%----------------------------------------------------------------------------------------
\subsection{Compressing}  % 30 to 45 min
%----------------------------------------------------------------------------------------
Using only command-line in your Linux shell,
\begin{enumerate}
    \item Create an empty Git repository in your root directory
          (if you have accidentally already created a Git repository in your root directory, delete it before).
    \item Check that Git is aware of your 2 files but does not track them yet.
    \item Check that no object is stored yet in the \textbf{objects} subdirectory of your Git repository.
    \item Create a directory inside the \textbf{objects} subdirectory of your Git repository,
          whose name is the first two characters of the SHA-1 hash computed in the previous exercise.
    \item Install the \href{https://en.wikipedia.org/wiki/QPDF}{QPDF} free command-line program. \\
          Part of this program is the \textbf{zlib-flate} command that compress and uncompress files
          using the \textbf{deflate} algorithm.
    \item Create a file inside the directory that you have just created,
          whose content is the deflate compression (level 1) of your second file
          and whose name is the last 38 characters of the SHA-1 hash computed in the previous exercise.
    \item Check that Git successfully considers this file as one of its inner object. \\
        \textit{hint: You can use the different flags of the \textbf{git cat-file} plumbing command.}
    \item Backup your Git repository and create a new one.
    \item Stage your first file in Git and check that its name and content are identical to yours.
    \item Create another text file whose content is 100 lines of ``Hello Mister i'' (i varying from 1 to 100).
    \item Stage this new file in Git
          and check that the compression ratio on this second example is better than on the first one.
\end{enumerate}

\ifdefined\answer
\begin{minted}{bash}
    rm -r .git
    git init
    git status
    find .git/objects/ -type f
    sudo apt install qpdf
    mkdir .git/objects/55
    zlib-flate -compress=1 < file1_git > .git/objects/55/7db03de997c86a4a028e1ebd3a1ceb225be238
    git cat-file -p 557db03
    git cat-file -t 557db03
    git cat-file -s 557db03
    mv .git/ .git.bu
    git init
    git add file1.txt
    diff .git/objects/55/7db03de997c86a4a028e1ebd3a1ceb225be238 .git.bu/objects/55/7db03de997c86a4a028e1ebd3a1ceb225be238
    for i in `seq 1 100`; do echo "Hello Mister $i" >> file2.txt; done
    git add file2.txt
    find -type f -ls | grep -v sample
\end{minted}
\fi

%----------------------------------------------------------------------------------------
\subsection{Local and remote backup}  % 15 to 25 min
%----------------------------------------------------------------------------------------

Using only command-line in your Linux shell,
\begin{enumerate}
    \item Commit the two files that must be in your staging area.
    \item Check that more objects have been saved in your Git repository.
    \item Check that a commit exists in your Git repository. \\
        \textit{hint: You can use the \textbf{git log} porcelain command.}
    \item Walk Git inner directed acyclic graph (DAG) from your commit to your two file contents. \\
        \textit{hint: You can use several times the \textbf{git cat-file} plumbing command.} \\
        \textit{hint: If you are a teaser you can try combining
        \textbf{awk}, \textbf{print}, \textbf{xargs} and \textbf{head} commands.}
    \item Suppress your two files.
    \item Use Git to restore your files.
    \item Backup your Git repository elsewhere \\
          (pretending a copy exists on another colleague's computer or on a remote server).
    \item Suppress your root directory, create a new empty one and use your backup to restore everything.
\end{enumerate}

\ifdefined\answer
\begin{minted}{bash}
    git commit -m "initial commit"
    tree .git/objects/
    git log
    # ---------------- hard way ----------------
    git log --oneline | awk '{print $1}' | xargs -I{} git cat-file -p {}
    git log --oneline | awk '{print $1}' | xargs -I{} git cat-file -p {} | head -n 1 | awk '{print $2}' | xargs -I{} git cat-file -p {}
    # ---------------- soft way ----------------
    git cat-file -p 7d0c488
    git cat-file -p 8407e2e
    # ------------------------------------------
    rm *
    git restore .
    cp .git ..
    cd ..
    rm -rf exercice_root
    mkdir exercice_root
    cd exercice_root/
    mv ../.git .
    git checkout HEAD
    git restore .
\end{minted}
\fi

%----------------------------------------------------------------------------------------
\subsection{Diff}  % 10 to 15 min
%----------------------------------------------------------------------------------------

Using only command-line in your Linux shell,
\begin{enumerate}
    \item Add one line to your first ``Hello World'' file.
    \item Delete the last 10 lines of your second ``Hello Mister i'' file. \\
        \textit{hint: You can use the \textbf{d} command and the \textbf{-i} flag of the \textbf{sed} Unix command.}
    \item Check that Git is aware of your changes
    \item Display the changes in your root directory since the last commit.
          You should see your two changes. \\
        \textit{hint: You can use the \textbf{git diff} Unix command.}
    \item Stage your first file only.
    \item Check that Git has correctly staged only one file.
    \item Display again the changes in your root directory since the last commit.
          Which file changes do you see ?
    \item Display the other file changes.
\end{enumerate}

\ifdefined\answer
\begin{minted}{bash}
    echo "This is so much fun" >> file1.txt
    sed -i '91,100d' file2.txt
    git status
    git add file1.txt
    git status
    git diff
    git diff --staged
\end{minted}
\fi

%----------------------------------------------------------------------------------------
\subsection{Undo}  % 10 to 15 min
%----------------------------------------------------------------------------------------

Using only command-line in your Linux shell,
\begin{enumerate}
    \item Unstage your first file
    \item Commit your two file changes directly, without staging them.
    \item Check your commit log history. Do you see your new commit ?
    \item Without affecting your Git repository, set your root directory state
          as of the snapshot of your first commit.
    \item Check your commit log history. You do not see all commits, do you ? How can you see all of them ?
    \item Return to the snapshot of your your last commit.
    \item Undo your second commit by adding a new commit that reverts it.
    \item Check the content of your root directory. Have your previous changes disappeard ?
    \item Check your commit log history. Do you see your revert commit ?
    \item Remove the last 2 commits from the history.
    \item Check the content of your root directory. Have your previous changes disappeard ?
    \item Check your commit log history. Have you lost the last 2 commits ?
\end{enumerate}

\ifdefined\answer
\begin{minted}{bash}
    git restore --staged file1.txt
    git commit -am "add interesting stuff"
    git log
    git checkout HEAD~
    git log --all --graph
    git checkout master
    git revert --no-edit HEAD
    git log --all --graph
    git reset --hard HEAD~2
    git log --all --graph
\end{minted}
\fi

%----------------------------------------------------------------------------------------
\subsection{Aliases}  % 5 to 10 min
%----------------------------------------------------------------------------------------

Using only command-line in your Linux shell,
\begin{enumerate}
    \item Create a ``\textbf{s}'' alias for the \textbf{git status} command.
    \item Create a ``\textbf{co}'' alias for the \textbf{git checkout} command.
    \item Create a ``\textbf{b}'' alias for the \textbf{git branch} command.
    \item Create a ``\textbf{ci}'' alias for the \textbf{git commit} command.
    \item Create a ``\textbf{dog}'' alias for the \textbf{git log --all --decorate --oneline --graph} command.
    \item Create a ``\textbf{dag}'' alias for the \textbf{git log --all --decorate --graph} command.
    \item Create a ``\textbf{list}'' alias for the \textbf{git diff-tree --no-commit-id --name-only -r} command.
    \item Create a ``\textbf{unstage}'' alias for the \textbf{git reset HEAD --} command.
    \item Create a ``\textbf{last}'' alias for the \textbf{git log -1 HEAD} command.
\end{enumerate}

\ifdefined\answer
\begin{minted}{bash}
    git config --global alias.s status
    git config --global alias.co checkout
    git config --global alias.b branch
    git config --global alias.ci commit
    git config --global alias.dog 'log --all --decorate --oneline --graph'
    git config --global alias.dag 'log --all --decorate --graph'
    git config --global alias.list 'diff-tree --no-commit-id --name-only -r'
    git config --global alias.unstage 'reset HEAD --'
    git config --global alias.last 'log -1 HEAD'
\end{minted}
\fi

\end{document}
