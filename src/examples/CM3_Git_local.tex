\input{CM_class.tex}

\begin{document}

\begin{frame}
   \titlepage
\end{frame}

\begin{frame}
   \tableofcontents
\end{frame}

\section{Linux}
\section{Git}

%----------------------------------------------------------------------------------------
\subsection{Version Control System}
%----------------------------------------------------------------------------------------

\begin{frame}\frametitle{When do you need a Version Control System (VCS) ?}
   When you have:
   \begin{itemize}
      \item Project directory
      \item With text files
      \item Living
      \item Precious
   \end{itemize}
   \vspace{1em}
   Typical examples: Software development project, Internship report
\end{frame}

\begin{frame}\frametitle{Manual Version Control}
   \begin{enumerate}
      \item Snapshot your project directory by copying it
      \item Suffix the copy with an increasing number (or a timestamp)
      \item Repeat this each time you think it is relevant
   \end{enumerate}
\end{frame}

\begin{frame}[fragile]\frametitle{Manual Version Control}
   \begin{lstlisting}[language=bash, style=bash, autogobble]
      $ ls -1
      Project
      Project_01
      Project_02
      Project_03
      Project_04
      Project_05
      Project_06
      Project_07
   \end{lstlisting}
\end{frame}

\begin{frame}\frametitle{Manual Version Control}
   \begin{enumerate}\setcounter{enumi}{3}
      \item Add a topic suffix each time you want to work on different topics
      \item Add a collaborator suffix each time you work with a different collaborator
   \end{enumerate}
\end{frame}

\begin{frame}[fragile]\frametitle{Manual Version Control}
   \begin{lstlisting}[language=bash, style=bash, autogobble]
      $ ls -1
      Project
      Project_01
      Project_02
      Project_03
      Project_04
      Project_05
      Project_06
      Project_07
      Project_07_Gui
      Project_07_Gui_Bernard
      Project_07_Gui_JeanPierre
      Project_07_Model
   \end{lstlisting}
\end{frame}

\begin{frame}\frametitle{Manual Version Control}
   \centering
   \Huge
   We can do better
\end{frame}

%----------------------------------------------------------------------------------------
\subsection{Introducing Git}
%----------------------------------------------------------------------------------------

\begin{frame}\frametitle{Git History}
   VCS created in 2005 to manage the Linux kernel \\
   as BitKeeper used since 2002 was proprietary and closed-source \\[2ex]

   Some of the goals of the new system were:
   \begin{itemize}
      \item Speed
      \item Simple design
      \item Strong support for non-linear development \\ (thousands of parallel branches)
      \item Fully distributed \\ (work offline, local branch, commit often, data protection)
      \item Able to handle large projects like the Linux kernel efficiently \\ (speed and data size)
   \end{itemize}
\end{frame}

\begin{frame}[fragile]\frametitle{Git Reference Manual}
   \begin{lstlisting}[language=bash, style=bash]
      $ git help git

      GIT(1)          Git Manual          GIT(1)
      NAME
            git - the stupid content tracker

      SYNOPSIS
            git [--version] [--help] [-C <path>] [-c <name>=<value>]
               [--exec-path[=<path>]] [--html-path] [--man-path] [--info-path]
               [-p|--paginate|-P|--no-pager] [--no-replace-objects] [--bare]
               [--git-dir=<path>] [--work-tree=<path>] [--namespace=<name>]
               [--super-prefix=<path>]
               <command> [<args>]

      DESCRIPTION
            Git is a fast, scalable, distributed revision control system
            with an unusually rich command set that provides both
            high-level operations and full access to internals.
   \end{lstlisting}
\end{frame}

%----------------------------------------------------------------------------------------
\subsection{First Git commands}
%----------------------------------------------------------------------------------------

\begin{frame}[fragile]\frametitle{Git configuration}
   \begin{center}
      \Huge git config \normalsize
   \end{center}
   \begin{lstlisting}[language=bash, style=bash]
      $ git config --global user.email "hoduche@yahoo.fr"
      $ git config --global user.name "Henri-Olivier Duche"
   \end{lstlisting}
   Saved in .gitconfig text file in user's home directory:
   \begin{lstlisting}[language=bash, style=bash, autogobble]
      $ cat ~/.gitconfig
      [user]
            email = hoduche@yahoo.fr
            name = Henri-Olivier Duche
   \end{lstlisting}
\end{frame}

\begin{frame}[fragile]\frametitle{Create a Git repository}
   \begin{center}
      \Huge git init \normalsize
   \end{center}
   \begin{lstlisting}[language=bash, style=bash]
      $ git init
      Initialized empty Git repository in /mnt/c/esilv/project/.git/
   \end{lstlisting}
   This creates a Git repository inside your directory \\
   i.e. a new subdirectory named ``.git''
   that contains \textbf{everything}
\end{frame}

\begin{frame}[fragile]\frametitle{Git repository status}
   \begin{center}
      \Huge git status \normalsize
   \end{center}
   \begin{lstlisting}[language=bash, style=bash, autogobble]
      $ echo "Hello World" > file1.txt
      $ git status
      On branch master

      No commits yet

      Untracked files:
        (use "git add <file>..." to include in what will be committed)
              file1.txt

      nothing added to commit but untracked files present (use "git add" to track)
   \end{lstlisting}
\end{frame}

\begin{frame}[fragile]\frametitle{Save a file in Git}
   \begin{center}
      \Huge git add \normalsize
   \end{center}
   \begin{lstlisting}[language=bash, style=bash]
      $ git add file1.txt
      $ git status
      On branch master

      No commits yet

      Changes to be committed:
        (use "git rm --cached <file>..." to unstage)
              new file:   file1.txt
   \end{lstlisting}
\end{frame}

%----------------------------------------------------------------------------------------
\subsection{Inside Git filesystem}
%----------------------------------------------------------------------------------------

\begin{frame}\frametitle{The Law of Leaky Abstractions}
   \begin{quote}
      ``All non-trivial abstractions, to some degree, are leaky'' \\
   \end{quote}
   \vspace{-1em}
   \begin{flushright}
      \href{https://www.joelonsoftware.com/2002/11/11/the-law-of-leaky-abstractions/}
      {\textcolor{DarkBlue}{Joel Spolsky - Stack Overflow website founder}}
   \end{flushright}
\end{frame}

\begin{frame}\frametitle{You'd better learn the ``Plumbing'' not just the ``Porcelain''}
   \begin{minipage}{0.48\linewidth}
      \includegraphics[width=\textwidth]{../images/git.png} \\
      \centering\href{https://xkcd.com/1597/}{\textcolor{DarkBlue}{xkcd}}
   \end{minipage}
   \begin{minipage}{0.48\linewidth}
      \begin{quote}
         ``All abstractions leak, and the only way to deal with the leaks competently is
         to learn about how the abstractions work and what they are abstracting.
         So the abstractions save us time working, but they don’t save us time learning''
      \end{quote}
      \begin{flushright}
         Joel Spolsky
      \end{flushright}
      \end{minipage}
\end{frame}

\begin{frame}[fragile]\frametitle{Look inside your Git repository}
   \begin{lstlisting}[language=bash, style=bash, autogobble]
      $ tree --charset=ascii .git
      .git
      |-- HEAD
      |-- branches
      |-- config
      |-- description
      |-- hooks
      |   `-- applypatch-msg.sample
      |-- index
      |-- info
      |   `-- exclude
      |-- objects
      |   |-- 55
      |   |   `-- 7db03de997c86a4a028e1ebd3a1ceb225be238
      |   |-- info
      |   `-- pack
      `-- refs
         |-- heads
         `-- tags

      10 directories, 7 files
   \end{lstlisting}
\end{frame}

\begin{frame}[fragile]\frametitle{Not that obvious}
   \begin{lstlisting}[language=bash, style=bash]
      $ cat .git/objects/55/7db03de997c86a4a028e1ebd3a1ceb225be238
   \end{lstlisting}
   \begin{center}
      \includegraphics[width=0.8\textwidth]{../images/compressed_content.png}
   \end{center}
   \vspace{1em}
   \begin{center}
      Cryptic file name + Cryptic file content
   \end{center}
\end{frame}

\begin{frame}[fragile]\frametitle{Lossless Compression}
   \begin{center}
      \Huge git cat-file -p \normalsize
   \end{center}
   \begin{lstlisting}[language=bash, style=bash]
      $ git cat-file -p 557db03
      Hello World
   \end{lstlisting}
   \vspace{1em}
   In Git filesystem, file content is compressed with Deflate lossless compression algorithm
   \footnote{https://zlib.net/feldspar.html}
\end{frame}

\begin{frame}[fragile]\frametitle{Blob}
   \begin{center}
      \Huge git cat-file -t \normalsize
   \end{center}
   \begin{lstlisting}[language=bash, style=bash]
      $ git cat-file -t 557db03
      blob
   \end{lstlisting}
   \vspace{1em}
   \begin{center}
      In Git filesystem, a file content is called a ``blob''
   \end{center}
\end{frame}

\begin{frame}[fragile]\frametitle{Cryptographic Hash}
   \begin{center}
      \Huge git hash-object \normalsize
   \end{center}
   \begin{lstlisting}[language=bash, style=bash]
      $ git hash-object file1.txt
      557db03de997c86a4a028e1ebd3a1ceb225be238
   \end{lstlisting}
   \vspace{1em}
   \begin{enumerate}
      \item A hash function converts any input to a fixed-size output
      \item A checksum adds avalanche effect
      \item A fingerprint adds very low probability of collision
      \item A cryptographic hash function adds reverse impossibility
   \end{enumerate}
   \vspace{1em}
   In Git filesystem, file content is hashed with SHA-1 algorithm. \\
   A cryptographic hash function that produces 40 hexadecimal digits output: \\
   2 for the directory / 38 for the file
\end{frame}

\begin{frame}[fragile]\frametitle{Display your Git repository as a Graph}
   \begin{center}
      \Huge git graph \normalsize
   \end{center}
   \begin{lstlisting}[language=bash, style=bash]
      $ git graph -n bt 2>/dev/null
      2021_01_11_00_45_11_git_graph.dot.pdf saved in /mnt/c/esilv/project/.gitGraph
   \end{lstlisting}
   \vspace{1em}
   \begin{center}
      \includegraphics[scale=1]{../images/2021_01_11_00_45_11_git_graph.dot-crop.pdf}
   \end{center}
   \vspace{1em}
   \begin{itemize}
      \item Git-graph is an \href{https://github.com/hoduche/git-graph}
   {\textcolor{DarkBlue}{open source Git plugin}} written in Python
      \item Alternatively, use a \href{https://marketplace.visualstudio.com/items?itemName=mhutchie.git-graph}
   {\textcolor{DarkBlue}{git-graph extension in VSCode}}
   \end{itemize}
\end{frame}

\begin{frame}[fragile]\frametitle{Save your work in Git}
   \begin{center}
      \Huge git commit \normalsize
   \end{center}
   \begin{lstlisting}[language=bash, style=bash]
      $ git commit -m "initial commit"
      [master (root-commit) 575bce5] initial commit
       1 file changed, 1 insertion(+)
       create mode 100644 file1.txt
   \end{lstlisting}
\end{frame}

\begin{frame}[fragile]\frametitle{Check your Git repository}
   \begin{minipage}{0.68\linewidth}
      \begin{lstlisting}[language=bash, style=bash, autogobble]
         $ tree --charset=ascii .git
         .git
         |-- COMMIT_EDITMSG
         |-- HEAD
         |-- branches
         |-- config
         |-- description
         |-- hooks
         |   `-- applypatch-msg.sample
         |-- index
         |-- info
         |   `-- exclude
         |-- logs
         |   |-- HEAD
         |   `-- refs
         |       `-- heads
         |           `-- master
         |-- objects
         |   |-- 30
         |   |   `-- a0b0faedd574340629413656ab79d82b588d96
         |   |-- 55
         |   |   `-- 7db03de997c86a4a028e1ebd3a1ceb225be238
         |   |-- 57
         |   |   `-- 5bce585ef620a4a6e9395c838f923a26705e7d
         |   |-- info
         |   `-- pack
         `-- refs
             |-- heads
             |   `-- master
             `-- tags

         15 directories, 13 files
      \end{lstlisting}
   \end{minipage}
   \begin{minipage}{0.28\linewidth}
      \centering
      \includegraphics[scale=0.75]{../images/short_graph.dot-crop.pdf}
   \end{minipage}
\end{frame}

\begin{frame}[fragile]\frametitle{Tree and commit}
   \begin{minipage}{0.68\linewidth}
      \begin{lstlisting}[language=bash, style=bash, autogobble]
         $ git cat-file -t 30a0b0f
         tree
         $ git cat-file -p 30a0b0f
         100644 blob 557db03de997c86a4a028e1ebd3a1ceb225be238    file1.txt
         $ git cat-file -t 575bce5
         commit
         $ git cat-file -p 575bce5
         tree 30a0b0faedd574340629413656ab79d82b588d96
         author Henri-Olivier Duche <hoduche@yahoo.fr> 1610576538 +0100
         committer Henri-Olivier Duche <hoduche@yahoo.fr> 1610576538 +0100

         initial commit
      \end{lstlisting}
      \vspace{1em}
      In Git filesystem:
      \begin{itemize}
         \item A directory is called a ``tree'' and points to blob(s) and other tree(s)
         \item A snapshot is called a ``commit'' and points to a tree and other commit(s)
      \end{itemize}
   \end{minipage}
   \begin{minipage}{0.28\linewidth}
      \centering
      \includegraphics[scale=0.75]{../images/short_graph.dot-crop.pdf}
   \end{minipage}
\end{frame}

\begin{frame}[fragile]\frametitle{Save more content in Git}
   \begin{lstlisting}[language=bash, style=bash, autogobble]
      $ mkdir folder
      mkdir: created directory 'folder'
      $ cp file1.txt folder/file2.txt
      'file1.txt' -> 'folder/file2.txt'
      $ echo "Hello World 3" > file3.txt
      $ git add .
      $ git commit -m "add intersting stuff"
      [master e5de582] add intersting stuff
         2 files changed, 2 insertions(+)
         create mode 100644 file3.txt
         create mode 100644 folder/file2.txt
   \end{lstlisting}
\end{frame}

\begin{frame}[fragile]\frametitle{Save more content in Git}
   \begin{minipage}{0.20\linewidth}
      \includegraphics[scale=0.5]{../images/short_graph.dot-crop.pdf}
   \end{minipage}
   \begin{minipage}{0.28\linewidth}
      \begin{lstlisting}[language=bash, style=bash, autogobble]
         $ tree
         .
         |-- file1.txt
         |-- file3.txt
         `-- folder
            `-- file2.txt

         1 directory, 3 files
      \end{lstlisting}
   \end{minipage}
   \begin{minipage}{0.48\linewidth}
      \begin{flushright}
      \includegraphics[scale=0.5]{../images/2021_01_14_10_14_08_git_graph.dot-crop.pdf}
      \end{flushright}
   \end{minipage}
\end{frame}

\begin{frame}[fragile]\frametitle{HEAD and master branch}
   \begin{minipage}{0.52\linewidth}
      \begin{lstlisting}[language=bash, style=bash, autogobble]
         $ cat .git/HEAD
         ref: refs/heads/master
         $ cat .git/refs/heads/master
         e5de5824bb151db5d2761ff576cbbeb8eb50959b
      \end{lstlisting}
      \vspace{1em}
      In Git filesystem:
      \begin{itemize}
         \item master is the default ``branch'' and points to the latest commit (snapshot)
         \item HEAD is a reference that points to the current branch
      \end{itemize}
   \end{minipage}
   \begin{minipage}{0.44\linewidth}
      \begin{flushright}
      \includegraphics[scale=0.5]{../images/2021_01_14_00_26_30_git_graph.dot-crop.pdf}
      \end{flushright}
   \end{minipage}
\end{frame}

\begin{frame}[fragile]\frametitle{Restore a snapshot}
   \begin{center}
      \Huge git checkout \normalsize
   \end{center}
   \begin{minipage}{0.48\linewidth}
      \begin{lstlisting}[language=bash, style=bash, autogobble]
         $ git checkout 575bce5
         Note: switching to '575bce5'.

         You are in 'detached HEAD' state.
         ...

         HEAD is now at 575bce5 initial commit
         $ tree --charset=ascii
         .
         `-- file1.txt

         0 directories, 1 file
      \end{lstlisting}
      \vspace{1em}
      At checkout, Git walks the \\
      directed acyclic graph (DAG) from the commit to recreate the corresponding snapshot
   \end{minipage}
   \begin{minipage}{0.48\linewidth}
      \begin{flushright}
         \includegraphics[scale=0.5]{../images/2021_01_14_11_10_23_git_graph.dot-crop.pdf}
      \end{flushright}
   \end{minipage}
\end{frame}

\begin{frame}[fragile]\frametitle{See your current changes}
   \begin{center}
      \Huge git diff \normalsize
   \end{center}
   \begin{lstlisting}[language=bash, style=bash, autogobble]
      $ vi file3.txt
      $ vi file1.txt

      $ git diff
      diff --git a/file1.txt b/file1.txt
      index 557db03..fe0dbd9 100644
      --- a/file1.txt
      +++ b/file1.txt
      @@ -1 +1,2 @@
       Hello World
      +I add a little information here
      diff --git a/file3.txt b/file3.txt
      index 22c0dee..efad9e7 100644
      --- a/file3.txt
      +++ b/file3.txt
      @@ -1 +1,2 @@
       Hello World 3
      +And a piece of information there
   \end{lstlisting}
\end{frame}

\begin{frame}[fragile]\frametitle{Staging area (a.k.a. index)}
   \begin{minipage}{0.63\linewidth}
      \begin{small}
      \begin{enumerate}
         \setlength\itemsep{1em}
         \item Modified: File changed locally
         \item Staged: File selected for next commit
         \item Committed: File saved in local .git
      \end{enumerate}
      \end{small}
   \end{minipage}
   \begin{minipage}{0.33\linewidth}
      \includegraphics[scale=0.18]{../images/staging.png}
   \end{minipage}
   \begin{lstlisting}[language=bash, style=bash, autogobble]
      $ git status
      On branch master
      Changes not staged for commit:
      (use "git add <file>..." to update what will be committed)
      (use "git restore <file>..." to discard changes in working directory)
            modified:   file1.txt
            modified:   file3.txt
      no changes added to commit (use "git add" and/or "git commit -a")

      $ git add file1.txt
      $ git status
      On branch master
      Changes to be committed:
      (use "git restore --staged <file>..." to unstage)
            modified:   file1.txt
      Changes not staged for commit:
      (use "git add <file>..." to update what will be committed)
      (use "git restore <file>..." to discard changes in working directory)
            modified:   file3.txt
   \end{lstlisting}
\end{frame}

\begin{frame}[fragile]\frametitle{See your current changes}
   \begin{center}
      \Huge git diff --staged\normalsize
   \end{center}
   \vspace{1em}
   \begin{minipage}{0.44\linewidth}
      \centering
      staging area\\
      vs\\
      .git repository
      \begin{lstlisting}[language=bash, style=bash, autogobble]
         $ git diff --staged
         diff --git a/file1.txt b/file1.txt
         index 557db03..fe0dbd9 100644
         --- a/file1.txt
         +++ b/file1.txt
         @@ -1 +1,2 @@
          Hello World
         +I add a little information here
      \end{lstlisting}
   \end{minipage}
   \hfill
   \begin{minipage}{0.44\linewidth}
      \centering
      working directory\\
      vs\\
      .git repository
      \begin{lstlisting}[language=bash, style=bash, autogobble]
         $ git diff
         diff --git a/file3.txt b/file3.txt
         index 22c0dee..efad9e7 100644
         --- a/file3.txt
         +++ b/file3.txt
         @@ -1 +1,2 @@
         Hello World 3
         +And a piece of information there
      \end{lstlisting}
   \end{minipage}
\end{frame}

\begin{frame}[fragile]\frametitle{Interactive staging}
   \begin{center}
      \Huge git add --interactive\normalsize
   \end{center}
   \vspace{1em}
   \begin{lstlisting}[language=bash, style=bash, autogobble]
      $ git add --interactive
               staged     unstaged path
      1:        +1/-0      nothing file1.txt
      2:    unchanged        +1/-0 file3.txt

      *** Commands ***
      1: status       2: update       3: revert       4: add untracked
      5: patch        6: diff         7: quit         8: help
      What now> u
               staged     unstaged path
      1:    unchanged        +1/-0 file3.txt
      Update>> 1
               staged     unstaged path
      * 1:    unchanged        +1/-0 file3.txt
      Update>>
   \end{lstlisting}
\end{frame}

\begin{frame}[fragile]\frametitle{Interactive staging}
   \begin{center}
      \Huge git add --patch\normalsize
   \end{center}
   \vspace{1em}
   It is possible to stage certain parts of a file\footnote{\url{https://nuclearsquid.com/writings/git-add/}}
   \vspace{1em}
   \begin{lstlisting}[language=bash, style=bash, autogobble]
      $ git add --patch file3.txt
      diff --git a/file3.txt b/file3.txt
      index 22c0dee..efad9e7 100644
      --- a/file3.txt
      +++ b/file3.txt
      @@ -1 +1,2 @@
      Hello World 3
      +And a piece of information there
      (1/1) Stage this hunk [y,n,q,a,d,e,?]?
   \end{lstlisting}
\end{frame}

\begin{frame}[fragile]\frametitle{Navigate between commits}
   \begin{itemize}
      \item \textbf{HEAD\~{}}: 1 commit older than HEAD
      \item \textbf{HEAD\^{}1}: 1 commit older than HEAD (first parent)
      \item \textbf{HEAD\^{}2}: 1 commit older than HEAD (second parent if HEAD was a merge commit)
      \item \textbf{HEAD\~{}2}: 2 commits older than HEAD
   \end{itemize}
   \vspace{1em}
   \begin{lstlisting}[language=bash, style=bash, autogobble]
      $ git diff HEAD~ HEAD
      diff --git a/file3.txt b/file3.txt
      new file mode 100644
      index 0000000..22c0dee
      --- /dev/null
      +++ b/file3.txt
      @@ -0,0 +1 @@
      +Hello World 3
      diff --git a/folder/file2.txt b/folder/file2.txt
      new file mode 100644
      index 0000000..557db03
      --- /dev/null
      +++ b/folder/file2.txt
      @@ -0,0 +1 @@
      +Hello World
   \end{lstlisting}
\end{frame}

\begin{frame}\frametitle{Undo work}
   \begin{itemize}
      \item \textbf{git reset <commit>}: removes commits from the branch (changes the commit history)
      \item \textbf{git revert <commit>}: creates a new commit that reverts an existing commit
      \item \textbf{git restore <file>}: discard changes in working directory (which have not been added to the staging zone)
      \item \textbf{git restore --staged <file>}: unstage file
   \end{itemize}
\end{frame}

% HERE
% ed2a4f88e8d361ed1d41032d126c37b87e301a00 (HEAD -> master) Commit that should dissapear
% a860791c6535a6bbd91f9ae1f82ac9f30a4f657f Init
\begin{frame}[fragile]\frametitle{Navigate between commits}
      \textbf{git reset a8607}: 1 commit older than HEAD
   \vspace{1em}
   \begin{lstlisting}[language=bash, style=bash, autogobble]
      $ git diff HEAD~ HEAD
      diff --git a/file3.txt b/file3.txt
      new file mode 100644
      index 0000000..22c0dee
      --- /dev/null
      +++ b/file3.txt
      @@ -0,0 +1 @@
      +Hello World 3
      diff --git a/folder/file2.txt b/folder/file2.txt
      new file mode 100644
      index 0000000..557db03
      --- /dev/null
      +++ b/folder/file2.txt
      @@ -0,0 +1 @@
      +Hello World
   \end{lstlisting}
\end{frame}

\begin{frame}\frametitle{Undo work - bring back content}
   \begin{itemize}
      \item \textbf{git checkout 156ad file1.txt}: Bring back a file from a specific version
      \item \textbf{git checkout 156ad file1.txt -p}: Bring back a file portion from a specific version
      \footnote{\url{https://www.youtube.com/watch?v=ipav1TCV8BI}}
      \footnote{\url{https://www.youtube.com/watch?v=MIFQwHlEI9o}}
   \end{itemize}
\end{frame}


\begin{frame}[fragile]\frametitle{Ignore files}
   Two main ways to ignore files:
   \begin{itemize}
      \item \textbf{.gitignore}: commited in your project and visible by others
      \item \textbf{.git/info/exclude}: not commited and not visible by others
   \end{itemize}
   \vspace{1em}
   Both use glob patterns to define ignored files:
   \begin{minipage}{0.44\linewidth}
      \begin{lstlisting}[language=bash, style=bash, autogobble]
         $ cat .gitignore
         logs/
         *.pdf
         *.aux
         *.log
      \end{lstlisting}
   \end{minipage}
   \hfill
   \begin{minipage}{0.44\linewidth}
      \begin{lstlisting}[language=bash, style=bash, autogobble]
         $ cat .git/info/exclude
         .vscode/
         *.o
         *.a
         *~
      \end{lstlisting}
   \end{minipage}
   \\[2em]
   There is a \href{https://github.com/github/gitignore}
   {\textcolor{DarkBlue}{collection of .gitignore templates}}
   on GitHub.
\end{frame}

\begin{frame}[fragile]\frametitle{Git Aliases}
   Feature that can make your Git experience simpler, easier, and more familiar
   \begin{itemize}
      \item Give shorcut to commands
      \item Create your own commands
   \end{itemize}
   \vspace{0.5em}
   Examples:
   \begin{lstlisting}[language=bash, style=bash, autogobble]
      $ git config --global alias.s status
      $ git config --global alias.co checkout
   \end{lstlisting}
   Saved in .gitconfig text file in user's home directory:
   \begin{lstlisting}[language=bash, style=bash, autogobble]
      $ cat ~/.gitconfig
      [user]
            email = hoduche@yahoo.fr
            name = Henri-Olivier Duche
      [alias]
            s = status
            co = checkout
   \end{lstlisting}
   Also achievable from \~{}/.bashrc in Linux
\end{frame}

\begin{frame}[fragile]\frametitle{Interesting Git Aliases}
   \begin{lstlisting}[language=bash, style=bash, autogobble]
      $ git config --global alias.s status
      $ git config --global alias.co checkout
      $ git config --global alias.b branch
      $ git config --global alias.ci commit
      $ git config --global alias.dog 'log --all --decorate --oneline --graph'
      $ git config --global alias.dag 'log --all --decorate --graph'
      $ git config --global alias.list 'diff-tree --no-commit-id --name-only -r'
      $ git config --global alias.unstage 'reset HEAD --'
      $ git config --global alias.last 'log -1 HEAD'
   \end{lstlisting}
\end{frame}

%----------------------------------------------------------------------------------------
\section{Python}

\end{document}
