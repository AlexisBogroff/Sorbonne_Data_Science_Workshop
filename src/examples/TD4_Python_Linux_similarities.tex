\input{TD_class.tex}
\setcounter{section}{3}

\begin{document}

%----------------------------------------------------------------------------------------
\section{Linux, Git and Python}
%----------------------------------------------------------------------------------------

%----------------------------------------------------------------------------------------
\subsection{Working Directory}  % 15 to 25 min
%----------------------------------------------------------------------------------------

Using only command-line in your Linux shell,
\begin{enumerate}
    \item Create an empty working directory called ``td4''.
    \item Initialize a Git repository in it.
    \item Install the Linux python3-pip package using your Linux package manager.
    \item Install the VirtualEnv Python package using pip3.
    \item Create a Python virtual environment called ``.env''. \\
          Do you see the change in your working directory ?
    \item Activate your virtual environment. \\
          Do you see the change in your prompt ?
    \item List the Python packages installed in your virtual environment.
    \item Does Git want you to commit something ? \\
          Do you think it is a good thing ? \\
          \textit{hint: You can find templates at \href{https://github.com/github/gitignore}
          {https://github.com/github/gitignore}}
    \item Create a .gitignore file to tell Git which files should be untracked.
    \item Does Git want you to commit something ? \\
          Do you think it is a good thing this time ?
    \item Do your first commit and check that Git is happy now.
\end{enumerate}

\ifdefined\answer
\begin{minted}{bash}
    mkdir td4
    cd td4
    git init
    sudo apt update
    sudo apt install python3-pip
    pip3 install virtualenv --force-reinstall
    python3 -m venv .env
    ls -la
    source .env/bin/activate
    pip list
    git status
    echo ".env/" > .gitignore
    git status
    git add .gitignore
    git commit -m "Initial commit"
    git status
\end{minted}
\fi

%----------------------------------------------------------------------------------------
\subsection{Python Script}  % 15 to 25 min
%----------------------------------------------------------------------------------------

Back to the Domesday Book, the greatest medieval census. \\
It lists the manors (private properties) in every place of every county in England
in the years 1066 and 1086, before and after the Norman conquest. \\
\href{https://opendomesday.org/}{OpenDomesday} presents it in a
modern-human-readable website,
as well as a RESTful web \textit{application programming interface} API:
\href{https://opendomesday.org/api/}{https://opendomesday.org/api/}

\begin{enumerate}
    \item Install the Python package \href{https://github.com/psf/requests}{Requests}
          using pip. \\
          \textit{info: Requests is a simple, yet elegant HTTP library
          and the de facto standard for querying RESTful web API.}
    \item Create a Python script that returns the list of all place ids in Derbyshire.
          \textit{hint: Look at the county structure inside the web API documentation} \\
          \textit{hint: As the web API returns JSON, no need to use regular expressions}
    \item Commit your changes in Git
\end{enumerate}

\ifdefined\answer
\begin{minted}{bash}
    pip install requests
    vi script.py
\end{minted}
\inputminted[]{python}{TD4_Python_Linux_similarities_answers_python_script.py}
\begin{minted}{bash}
    git add .
    git commit -m "retrieve all place ids in Derbyshire"
\end{minted}
\fi

%----------------------------------------------------------------------------------------
\subsection{Python Module}  % 15 to 25 min
%----------------------------------------------------------------------------------------

\begin{enumerate}
    \item Create a Python module with a \textbf{get\_manor\_ids} function
          that takes a place id as parameter and returns the list of manors.
    \item Check that calling your module does not produce any output.
    \item To test your module, open a python interpreter and call your function with
          the first place id from Derbyshire.
    \item Add a \textbf{if \_\_name\_\_ == '\_\_main\_\_':} block with your previous test,
          at the end of your module, to make it usable as a script.
    \item Check that calling your module now does produce an output.
    \item Commit your changes in Git
\end{enumerate}

\ifdefined\answer
\begin{minted}{bash}
    vi module.py
\end{minted}
\inputminted[firstline=1,lastline=11]{python}{TD4_Python_Linux_similarities_answers_python_module.py}
\begin{minted}{bash}
    python module.py
    python
    >>> import module
    >>> print(module.get_manor_ids(1036))
    [13038]
\end{minted}
\inputminted[]{python}{TD4_Python_Linux_similarities_answers_python_module.py}
\begin{minted}{bash}
    python module.py
    [13038]
    git add .
    git commit -m "retrieve all place ids in Derbyshire"
\end{minted}
\fi

%----------------------------------------------------------------------------------------
\subsection{Python Program}  % 15 to 25 min
%----------------------------------------------------------------------------------------

\begin{enumerate}
    \item Enrich your module to get all manors in all places in Derbyshire.
    \item Retrieve the geld paid and total ploughs owned by all those manors.
    \item Create a Pandas DataFrame with the same information.
    \item Use Pandas to compute the sum of geld paid and total ploughs owned in Derbyshire.
    \item Add docstrings to your functions.
    \item Commit your changes in Git.
\end{enumerate}

\ifdefined\answer
\inputminted[]{python}{TD4_Python_Linux_similarities_answers_python_full.py}
\fi

\end{document}