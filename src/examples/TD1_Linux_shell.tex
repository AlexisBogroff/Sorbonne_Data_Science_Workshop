\documentclass{article}
\usepackage{ae,lmodern}
\usepackage[english]{babel}
\usepackage[utf8]{inputenc}
\usepackage[T1]{fontenc}

\usepackage{caption}
\captionsetup[figure]{labelformat=empty}

\PassOptionsToPackage{usenames,dvipsnames}{xcolor}
\usepackage{xcolor,colortbl}
\definecolor{DarkGrey}{HTML}{222222}
\definecolor{DarkBlue}{HTML}{004BA9}
\definecolor{DarkRed}{HTML}{CC1111}
\definecolor{DarkGreen}{HTML}{117711}
\definecolor{DarkOrange}{HTML}{CC7000}
\definecolor{LightGrey}{HTML}{DDDDDD}
\definecolor{LightBlue}{HTML}{F0F8FF}
\definecolor{codegreen}{rgb}{0,0.6,0}
\definecolor{codepurple}{rgb}{0.58,0,0.82}

\usepackage[cache=false]{minted}
\setminted[bash]{
   bgcolor=LightBlue,
   breaklines, breakanywhere,
   frame=single,
   autogobble
}
\usemintedstyle[python]{native}
\setminted[python]{
   bgcolor=black,
   breaklines, breakanywhere,
   autogobble
}

\usepackage{listings}
\usepackage{lstautogobble}
\lstdefinestyle{bash}{
    backgroundcolor=\color{DarkGrey},   
    commentstyle=\color{codegreen},
    keywordstyle=\color{magenta},
    numberstyle=\tiny\color{DarkGrey},
    stringstyle=\color{codepurple},
    basicstyle=\ttfamily\tiny\color{LightGrey},
    escapeinside={\%*}{*)},
    breakatwhitespace=false,         
    breaklines=true,                 
    captionpos=b,                    
    keepspaces=true,                 
    numbers=left,                    
    numbersep=5pt,                  
    showspaces=false,                
    showstringspaces=false,
    showtabs=false,
    showlines=false,
    tabsize=2
}

\usepackage{tikz}
\usetikzlibrary{calc,decorations.pathreplacing,arrows,arrows.meta,shapes,patterns, positioning}
\newcommand\BigLength{14.6em}
\newcommand\Height{2em}
\newcommand\Sep{0.6em}
\newcommand\Center{\BigLength*1/2}
\newcommand\BigBox{\BigLength+\Sep}
\newcommand\HalfBox{\BigLength*1/2-\Sep*1/4}
\newcommand\HalfLength{\BigLength*1/2-\Sep*5/4}
\newcommand\CenterL{\BigLength*1/4-\Sep*1/8}
\newcommand\CenterR{\BigLength*3/4+\Sep*1/8}
\tikzstyle{layer}=[rectangle,thick,text centered,
                     minimum height=\Height,minimum width=\BigLength]
\tikzstyle{short}=[rectangle,thick,text centered,
                     minimum height=\Height,minimum width=\HalfLength]
\tikzstyle{dibox}=[rectangle,thick,semitransparent,
                     minimum height=(\Height+\Sep)*2,minimum width=\BigBox]
\tikzstyle{vmbox}=[rectangle,thick,semitransparent,
                     minimum height=(\Height+\Sep)*3,minimum width=\HalfBox]
\tikzstyle{ctbox}=[rectangle,thick,semitransparent,
                     minimum height=(\Height+\Sep)*2,minimum width=\HalfBox]
\tikzstyle{vebox}=[rectangle,thick,semitransparent,
                     minimum height=(\Height+\Sep)*1,minimum width=\HalfBox]

\usepackage{hyperref}
\usepackage{grffile}

\renewcommand{\thesection}{TD\arabic{section}/5:}
\renewcommand{\thesubsection}{Exercise \arabic{subsection}:}


\setcounter{section}{0}

\begin{document}
%----------------------------------------------------------------------------------------
%========================================================================================
\section{Use basic Linux commands}
%========================================================================================
%----------------------------------------------------------------------------------------


%****************************************************************************************
\subsection{Create, Rename, copy, move, delete}
%****************************************************************************************


\begin{enumerate}
    \item Go to your home directory
    \item Create a folder \textit{linux\_ex\_1}
    \item Make sure that you are in the right location
    \item Create an empty text file named \textit{[first\_name]\_[last\_name].txt} (e.g. \textit{alexis\_bogroff.txt})
    \item Create a folder \textit{notes}
    \item Move your text file into this folder
    \item Rename the text file by appending the current year \textit{[first\_name]\_[last\_name]\_[current\_year].txt}
    \item Make a copy of this folder, name it \textit{notes\_2022}
    \item Delete the first folder using the verbose option
\end{enumerate}


\ifdefined\answer
\begin{minted}{bash}
    cd  # This goes to the home
    mkdir linux_ex_1
    cd linux_ex_1
    pwd
    echo "" > alexis_bogroff.txt
    mkdir notes
    mv alexis_bogroff.txt notes/alexis_bogroff.txt
    mv notes/alexis_bogroff.txt notes/alexis_bogroff_2022.txt
    cp -r notes notes_2022
    rm -rv notes
\end{minted}
\fi


%****************************************************************************************
\subsection{Create and run a script}
%****************************************************************************************

\begin{enumerate}
    \item Create a script \textit{script\_1.sh} in the folder \textit{linux\_ex\_1}
    \item In the script, write the commands that would output the following:\\
        Script running please wait ...\\
        Done.
    \item Display the content of the script (using a command, not from an editor)
    \item Run the script
\end{enumerate}

\ifdefined\answer
\begin{minted}{bash}
    vim script_1.sh
        #!/bin/bash
        echo "Script running please wait ..."
        echo "Done."
    cat script_1.sh
    source script_1.sh
\end{minted}
\fi



%****************************************************************************************
\subsection{Accessing or modifying a file: permissions and root privilege}
%****************************************************************************************

%----------------------------------------------------------------------------------------
\subsubsection{Change the rights for accessing or modifying a file}
%----------------------------------------------------------------------------------------

\begin{enumerate}
    \item Create a file \textit{credentials} in the folder \textit{linux\_ex\_1}
    \begin{enumerate}
        \item Write any kind of (fake) personal information within the file
        \item Display the file content
        \item Display the current permissions
    \end{enumerate}
    \item Change the current permissions to: read only for all users
    \begin{enumerate}
        \item Display the new permissions
        \item Modify and save the file
        \item Display the file content
    \end{enumerate}
    \item Change the permissions back to read and write for all users
    \begin{enumerate}
        \item Display the new permissions
        \item Modify and save the file
        \item Display the file content
    \end{enumerate}
\end{enumerate}

\ifdefined\answer
\begin{minted}{bash}
    vim credentials
        id: cleverid
        pssd: veryprivatepassword
    cat credentials
    ls -l

    chmod 444 credentials
    ls -l
    vim credentials
    # You cannot modify the file as expected! (use :q! to escape from this trap!)
    cat credentials
    # Thus the content hasn't changed as expected!

    chmod 666 credentials
    ls -l
    vim credentials
    cat credentials
\end{minted}
\fi

\hfill \break
On the same file:
\begin{enumerate}
    \item Add the execute permission to the owner
    \begin{enumerate}
        \item Display the new permissions
    \end{enumerate}
    \item Remove the read permission to other users
    \begin{enumerate}
        \item Display the new permissions
    \end{enumerate}
    \item Change the permissions to read, write and execute for all users
    \begin{enumerate}
        \item Display the new permissions
    \end{enumerate}
\end{enumerate}

\ifdefined\answer
\begin{minted}{bash}
    chmod u+x credentials
    ls -l

    chmod o-r credentials
    ls -l

    chmod 777 credentials
    ls -l

    # See how it is cleaner to use absolute mode when changing permissions of all users, and to user symbolic mode for minimal change.
\end{minted}
\fi


%----------------------------------------------------------------------------------------
\subsubsection{Access root files}
%----------------------------------------------------------------------------------------

\begin{enumerate}
    \item Go to the root folder
    \item Create a file in root user mode named \textit{.private\_file}
    \begin{enumerate}
        \item Write some information in the file
        \item Display the file content
        \item Display all the files in the folder including hidden files
    \end{enumerate}
    \item Modify the file in normal user mode
    \begin{enumerate}
        \item Write some new information in the file
        \item Display the file content
    \end{enumerate}
    \item Modify the file in root user mode
    \begin{enumerate}
        \item Write some new information in the file
        \item Display the file content
    \end{enumerate}
    \item Change permissions to read, write and execute for all users
    \begin{enumerate}
        \item Modify the file content in normal user mode
        \item Display the file content
    \end{enumerate}
\end{enumerate}

\ifdefined\answer
\begin{minted}{bash}
    cd /.
    sudo vim .private_file
        This file can be modified with root privilege only
    cat .private_file
    ls -a

    vim .private_file
    # This cannot be edited withtout root privilege as expected
    cat .private_file

    sudo vim .private_file
        This file can be modified with root privilege only
        cat .private_file

    # See how changing the permissions require to have a root privilege.
    sudo chmod 777 .private_file
    vim .private_file
        This file can now be modified by any user!
    cat .private_file
\end{minted}
\fi

%----------------------------------------------------------------------------------------
\subsubsection{Change a file owner}
%----------------------------------------------------------------------------------------

\begin{enumerate}
    \item Change permissions of \textit{.private\_file} to read and write for all users, in normal user mode
    \item Set the new file owner as the current user
    \item Change permissions of \textit{.private\_file} to read and write for all users, in normal user mode
\end{enumerate}


\ifdefined\answer
\begin{minted}{bash}
    chmod 666 .private_file
    # See how it is still necessary to be a root user, even though the permissions have been changed

    sudo chown alexisbogroff .private_file

    chmod 666 .private_file
    ls -l .private_file
\end{minted}
\fi

%****************************************************************************************
\subsection{Extract data from a website}
%****************************************************************************************

The Domesday Book is the greatest medieval census.
It lists the manors (private properties) in every place of every county in England in the years 1066 and 1086, before and after the Norman conquest.
\href{https://opendomesday.org/}{OpenDomesday} presents it in a modern-human-readable website, as well as an {\it application programming interface} (API).

We will use this API to extract some data from our command-line shell.

{\bf The Internet is your friend, do not hesitate to ask his help to find a better command.}

%----------------------------------------------------------------------------------------
\subsubsection{curl}
%----------------------------------------------------------------------------------------

Check the data on https://opendomesday.org/api/, for instance

\begin{itemize}
    \item https://opendomesday.org/api/1.0/county/
    \item https://opendomesday.org/api/1.0/place/2346/
    \item https://opendomesday.org/api/1.0/manor/181/
\end{itemize}

Can you find other interesting URLs?

\ifdefined\answer
\begin{minted}{bash}
    # examples:
    # can try any manor id < 25000
    # https://opendomesday.org/api/1.0/manor/20385/
    # can also try other places, and from places get manors
\end{minted}
\fi

%----------------------------------------------------------------------------------------
\subsubsection{curl and grep}
%----------------------------------------------------------------------------------------

Let's try to get the ids for all the places in Derbyshire!

\ifdefined\answer
\inputminted[firstline=1,lastline=7]{bash}{TD1_Linux_shell_answers.sh}
\fi

%----------------------------------------------------------------------------------------
\subsubsection{curl, grep and for}
%----------------------------------------------------------------------------------------

Now that we have ids for all the places in Derbyshire, we can load all their details...

And from their details, we can list all the details of their \href{http://www.domesdaybook.net/domesday-book/structure-of-domesday-book/manor}{manors}.

Go grep the data!

{\bf You may write the raw data into a file to avoid downloading it everytime.}

\ifdefined\answer
\begin{minted}{bash}
    # first list all the places ids, easy as they are the only numerical entries
    grep -oP '\d+' derbyshire.json
    # and try load the manors data from a single place
    curl https://opendomesday.org/api/1.0/place/43196/ | grep -oP '\[(\{"id":\d+\})+]' | grep -oP '\d+'
    # now we can loop on it
\end{minted}
\inputminted[firstline=9,lastline=23]{bash}{TD1_Linux_shell_answers.sh}
\fi

%----------------------------------------------------------------------------------------
\subsubsection{curl, grep, for and sed}
%----------------------------------------------------------------------------------------

Now that we have a heap of raw data, we will extract the interesting parts.
In our case we want to count the \href{http://www.domesdaybook.net/domesday-book/data-terminology/taxation/tax-or-geld}{geld} paid by each manor
and compare it to the number of \href{http://www.domesdaybook.net/domesday-book/data-terminology/taxation/plough}{ploughs} it owns.

\begin{itemize}
    \item Can you find the corresponding json fields?
    \item Then you can list these numbers for each manor in Derbyshire.
    \item And format this in a proper {\it comma-separated values} (CSV) file.
\end{itemize}

\ifdefined\answer
\inputminted[firstline=25,lastline=31]{bash}{TD1_Linux_shell_answers.sh}
\fi

%----------------------------------------------------------------------------------------
\subsubsection{discover new commands}
%----------------------------------------------------------------------------------------

The CSV file you created could be loaded in Excel. But do you have one?

Use your search skills to find a way to sum values in a column and provide the final result.

\ifdefined\answer
\begin{minted}{bash}
    # google: https://www.google.com/search?client=firefox-b-d&q=bash+sum+csv+columns
    # stackoverflow: https://stackoverflow.com/questions/18683368/command-to-sum-2nd-colum-of-csv-file

    # for geld:
    awk -F"," '{print;x+=$2}END{print "Total " x}' manors.csv
    # for ploughs
    awk -F"," '{print;x+=$3}END{print "Total " x}' manors.csv
\end{minted}
\fi

%----------------------------------------------------------------------------------------
\subsubsection{Bonus}
%----------------------------------------------------------------------------------------

Install gnuplot, and generate a pretty graph from your CSV file.

%----------------------------------------------------------------------------------------
\subsubsection{Bonus}
%----------------------------------------------------------------------------------------

Use Vim to write a single bash script that does all of these actions.

\end{document}