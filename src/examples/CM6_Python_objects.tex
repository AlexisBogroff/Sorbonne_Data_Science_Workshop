\input{CM_class.tex}

\begin{document}

\begin{frame}
   \titlepage
\end{frame}

\begin{frame}
   \tableofcontents
\end{frame}

\section{Linux}
\section{Git}
\section{Python}

%----------------------------------------------------------------------------------------
\subsection{Python Objects}
%----------------------------------------------------------------------------------------

% explain the interest of OOP
% syntax of objects in python
% differences with objects in Java/.NET/JS/C++ (no force private, functions take self as parameter, standard functions in place of interfaces)

\begin{frame}[fragile]
\frametitle{Imperative Programming}
   \begin{minted}{python}
      sum = 1 + 1
      print(f"1 + 1 = {sum}")
   \end{minted}
\end{frame}

\begin{frame}[fragile]
\frametitle{Imperative Programming with control structures}
   \begin{minted}{python}
      for n in range(1, 10):
         sum = n + 1
         print(f"{n} + 1 = {sum}")
   \end{minted}
\end{frame}

\begin{frame}[fragile]
\frametitle{Imperative Programming with functions}
   \begin{minted}{python}
      def sqrt(number):
         """Compute square root of integer number"""
         ...
         return ...

      for n in range(1, 100):
         print(f"square root of {n} is {sqrt(n)}")
   \end{minted}
\end{frame}

\begin{frame}[fragile]
\frametitle{Imperative Programming with complex systems}
   \begin{minted}[fontsize=\scriptsize]{python}
      stock1 = dict(name="stock1", isin="FR010578277", orders=list(), status=OPEN)
      market = list(stock1, stock2, ...)

      def insert_order(market, stock_name, price quantity):
         stock = find_stock(market, stock_name)
         existing_order = find_order(stock.orders, price, quantity)
         if existing_order:
            raise "Already exist"
         new_order = add_order(stock.orders, price, quantity)
         check_order_executes(stock.orders, order)
         ...
   \end{minted}
   \begin{itemize}
      \item<2-> What if I need to index orders by id? add a validity to orders? handle auctions?
   \end{itemize}

\end{frame}

\begin{frame}
\frametitle{Imperative Programming}
   \begin{itemize}
      \item Strong separation between state (i.e. data) and instructions (i.e. changing data).

      \item Strong code coupling between data structure and functions.
      % \item Data structure cannot be easily changed.
      \item Hard to reason about for complex domains.
   \end{itemize}
\end{frame}

% \begin{frame}[fragile]
%    \frametitle{Strong code coupling between data structure and functions}
%    \begin{minted}{python}
%       # Because everything is an object, even a data structure.
%       # So a function can be accessed from the data itself
%       >> 'I love chocolate'.upper()
%       'I LOVE CHOCOLATE'
%       >> 12.54.__int__()
%       12
%       >> x = 1
%       >> x.__add__(3)
%       4
%    \end{minted}
% \end{frame}


\begin{frame}
\frametitle{Object Oriented Programming}
   \begin{itemize}
      \item Object contains both
            \begin{itemize}
               \item state (called fields or attributes)
               \item instructions (called methods or functions)
            \end{itemize}
      \item Classes define objects.
      \item A class \textit{encapsulate}s a concept.
   \end{itemize}
\end{frame}

\begin{frame}
\frametitle{Object Oriented Programming in Python}
   Do not forget the reference doc: \\
   https://docs.python.org/3/tutorial/classes.html
\end{frame}

\begin{frame}[fragile]
\frametitle{Object Oriented Programming in Python}
   \begin{minted}[fontsize=\scriptsize]{python}
      class MyClass:
         """A simple example class"""
         i = 12345 # an attribute of the class

         def say_hello(self):
            """A simple function in the class"""
            return 'hello world'

      myObject = MyClass()
      print(myObject.say_hello()) # hello world
      print(myObject.i) # 12345
   \end{minted}
\end{frame}

\begin{frame}[fragile]
\frametitle{Class Constructor}
   \begin{minted}[fontsize=\scriptsize]{python}
      class Order:
         def __init__(self, quantity, price, buy=True):
            self.quantity = quantity
            self.price = price
            self.buy = buy

         def is_sell(self):
            return not self.buy

      o = Order(5, 11.0)
      print(o.quantity) # 5
      print(o.is_sell()) # False
   \end{minted}
\end{frame}

\begin{frame}[fragile]
\frametitle{Python is permissive}
   \begin{minted}[fontsize=\scriptsize]{python}
      class Order:
         def __init__(self, quantity, price, buy=True):
            self.quantity = quantity
            self.price = price
            self.buy = buy

      o = Order(5, 11.0)
      print(o.quantity) # 5

      o.quantity = 10
      print(o.quantity) # 10
   \end{minted}
\end{frame}

\begin{frame}[fragile]
\frametitle{Python is really permissive}
   \begin{minted}[fontsize=\scriptsize]{python}
      class Order:
         def __init__(self, quantity, price, buy=True):
            self.__quantity = quantity
            self.price = price
            self.buy = buy

         def quantity(self):
            return self.__quantity if self.buy else -self.__quantity

      o = Order(5, 11.0)
      print(o.quantity()) # 5
      print(o.__quantity) # 5

      o.__quantity = 10
      print(o.quantity()) # 10
      print(o.__quantity) # 10
   \end{minted}
\end{frame}

\begin{frame}[fragile]
\frametitle{Python is really really permissive}
   \begin{minted}[fontsize=\scriptsize]{python}
      class Order:
         def __init__(self, quantity, price, buy=True):
            self.__quantity = quantity
            self.price = price
            self.buy = buy

      o = Order(5, 11.0)
      o.quantity = 10
      print(o.quantity) # 10
      print(o.__quantity) # 5
   \end{minted}
\end{frame}

\begin{frame}[fragile]
\frametitle{Class - Common functions - string representation}
   \begin{minted}[fontsize=\scriptsize]{python}
      class Order:
         def __init__(self, quantity, price):
            self.quantity = quantity
            self.price = price

         def __str__(self): # human-readable content
            return f"{self.quantity} @ {self.price}")

         def __repr__(self): # unambiguous representation of the object
            return f"Order({self.quantity}, {self.price})")

      o = Order(5, 11.0)
      print(o) # 5 @ 11.0
      print(str(o)) # 5 @ 11.0
      print(repr(o)) # Order(5, 11.0)
   \end{minted}
\end{frame}

\begin{frame}[fragile]
\frametitle{Class - Common functions - comparisons}
   \begin{minted}[fontsize=\scriptsize]{python}
      class Order:
         def __init__(self, quantity, price):
            self.quantity = quantity
            self.price = price

         def __eq__(self, other): # self == other
            return other and self.quantity == other.quantity and self.price == other.price

         def __lt__(self, other): # self < other
            return other and self.price < other.price

      o1 = Order(5, 11.0)
      o2 = Order(20, 10.0)
      print(o1 == o2) # False
      print(o1 < o2) # False
      print(o2 < o1) # True
      print(o1 > o2) # True
      print(o1 >= o2) # TypeError: '>=' not supported between instances of 'Order' and 'Order'
   \end{minted}
\end{frame}

\begin{frame}[fragile]
\frametitle{Class - Common functions - comparisons}
   \begin{minted}[fontsize=\scriptsize]{python}
      from functools import total_ordering

      @total_ordering
      class Order:
         def __init__(self, quantity, price):
            self.quantity = quantity
            self.price = price

         def __eq__(self, other): # self == other
            return other and self.quantity == other.quantity and self.price == other.price

         def __lt__(self, other): # self < other
            return other and self.price < other.price

      o1 = Order(5, 11.0)
      o2 = Order(20, 10.0)
      print(o1 == o2) # False
      print(o1 < o2) # False
      print(o2 < o1) # True
      print(o1 > o2) # True
      print(o1 >= o2) # True
   \end{minted}
\end{frame}


%----------------------------------------------------------------------------------------
\subsection{Order Book Example}
%----------------------------------------------------------------------------------------

\begin{frame}
\frametitle{What is an exchange order book?}
   \begin{itemize}
      \item Contains all orders on a financial instrument (stock, bond, future, option, etc.)
      \item Orders are usually sorted by execution order
         \begin{itemize}
            \item \textit{BUY:} from higher price to worst price
            \item \textit{SELL:} from lower price to worst price
         \end{itemize}
         First orders (with the \textit{best price}) are the first to execute. \\
         Most exchanges also prioritize by insertion order: first in, first out.
      \item Display can aggregate orders on the same limit (i.e. price level).
   \end{itemize}
\end{frame}

\begin{frame}
\frametitle{Trading sequence example}
   Start with an empty book on Orange
   \begin{table}
      \centering
      \resizebox{\textwidth}{!}{%
      \begin{tabular}{ r | r || l | l }
         \multicolumn{4}{c}{ Orange - FR0000133308 @ XPAR } \\
         \hline
         \multicolumn{2}{c}{ BUY } & \multicolumn{2}{c}{ SELL }  \\
         \hline
         \hspace{1cm} & \hspace{1cm} & \hspace{1cm} & \hspace{1cm} \\
      \end{tabular}
      } % end of scope of "\resizebox"  directive
   \end{table}
\end{frame}

\begin{frame}
\frametitle{Trading sequence example}
   Buy 10 stocks at 10.0
   \begin{table}
      \centering
      \resizebox{\textwidth}{!}{%
      \begin{tabular}{ r | r || l | l }
         \multicolumn{4}{c}{ Orange - FR0000133308 @ XPAR } \\
         \hline
         \multicolumn{2}{c}{ BUY } & \multicolumn{2}{c}{ SELL }  \\
         \hline
         {\bf 10} & {\bf 10.0} & & \\
         \hspace{1cm} & \hspace{1cm} & \hspace{1cm} & \hspace{1cm} \\
      \end{tabular}
      } % end of scope of "\resizebox"  directive
   \end{table}
\end{frame}

\begin{frame}
\frametitle{Trading sequence example}
   Sell 5 stocks at 11.0
   \begin{table}
      \centering
      \resizebox{\textwidth}{!}{%
      \begin{tabular}{ r | r || l | l }
         \multicolumn{4}{c}{ Orange - FR0000133308 @ XPAR } \\
         \hline
         \multicolumn{2}{c}{ BUY } & \multicolumn{2}{c}{ SELL }  \\
         \hline
         10 & 10.0 & {\bf 11.0} & {\bf 5} \\
         \hspace{1cm} & \hspace{1cm} & \hspace{1cm} & \hspace{1cm} \\
      \end{tabular}
      } % end of scope of "\resizebox"  directive
   \end{table}
\end{frame}

\begin{frame}
\frametitle{Trading sequence example}
   Buy 123 stocks at 10.5
   \begin{table}
      \centering
      \resizebox{\textwidth}{!}{%
      \begin{tabular}{ r | r || l | l }
         \multicolumn{4}{c}{ Orange - FR0000133308 @ XPAR } \\
         \hline
         \multicolumn{2}{c}{ BUY } & \multicolumn{2}{c}{ SELL }  \\
         \hline
         {\bf 123} & {\bf 10.5} & 11.0 & 5 \\
         10 & 10.0 & &  \\
         \hspace{1cm} & \hspace{1cm} & \hspace{1cm} & \hspace{1cm} \\
      \end{tabular}
      } % end of scope of "\resizebox"  directive
   \end{table}
\end{frame}

\begin{frame}
\frametitle{Trading sequence example}
   Sell 62 stocks at 11.0
   \begin{table}
      \centering
      \resizebox{\textwidth}{!}{%
      \begin{tabular}{ r | r || l | l }
         \multicolumn{4}{c}{ Orange - FR0000133308 @ XPAR } \\
         \hline
         \multicolumn{2}{c}{ BUY } & \multicolumn{2}{c}{ SELL }  \\
         \hline
         123 & 10.5 & 11.0 & 5 \\
         10 & 10.0 & {\bf 11.0} & {\bf 62} \\
         \hspace{1cm} & \hspace{1cm} & \hspace{1cm} & \hspace{1cm} \\
      \end{tabular}
      } % end of scope of "\resizebox"  directive
   \end{table}
   They appear below the best SELL order because they will execute after the previous SELL order.
\end{frame}

\begin{frame}
\frametitle{Trading sequence example}
   Sell 16 stocks at 10.6
   \begin{table}
      \centering
      \resizebox{\textwidth}{!}{%
      \begin{tabular}{ r | r || l | l }
         \multicolumn{4}{c}{ Orange - FR0000133308 @ XPAR } \\
         \hline
         \multicolumn{2}{c}{ BUY } & \multicolumn{2}{c}{ SELL }  \\
         \hline
         123 & 10.5 & {\bf 10.6} & {\bf 16} \\
         10 & 10.0 & 11.0 & 5 \\
          &  & 11.0 & 62 \\
         \hspace{1cm} & \hspace{1cm} & \hspace{1cm} & \hspace{1cm} \\
      \end{tabular}
      } % end of scope of "\resizebox"  directive
   \end{table}
\end{frame}

\begin{frame}
\frametitle{Trading sequence example}
   Buy 2 stocks at 10.6
   \begin{table}
      \centering
      \resizebox{\textwidth}{!}{%
      \begin{tabular}{ r | r || l | l }
         \multicolumn{4}{c}{ Orange - FR0000133308 @ XPAR } \\
         \hline
         \multicolumn{2}{c}{ BUY } & \multicolumn{2}{c}{ SELL }  \\
         \hline
         {\bf 2} & {\bf 10.6} & {\bf 10.6} & {\bf 16} \\
         123 & 10.5 & 11.0 & 5 \\
         10 & 10.0 & 11.0 & 62 \\
         \hspace{1cm} & \hspace{1cm} & \hspace{1cm} & \hspace{1cm} \\
      \end{tabular}
      } % end of scope of "\resizebox"  directive
   \end{table}
   BUY and SELL orders match $\rightarrow$ an execution happens! \\
   Execute 2 at 10.6...
\end{frame}

\begin{frame}
\frametitle{Trading sequence example}
   ...Execute 2 at 10.6
   \begin{table}
      \centering
      \resizebox{\textwidth}{!}{%
      \begin{tabular}{ r | r || l | l }
         \multicolumn{4}{c}{ Orange - FR0000133308 @ XPAR } \\
         \hline
         \multicolumn{2}{c}{ BUY } & \multicolumn{2}{c}{ SELL }  \\
         \hline
         123 & 10.5 & 10.6 & {\bf 14} \\
         10 & 10.0 & 11.0 & 5 \\
          & & 11.0 & 62 \\
         \hspace{1cm} & \hspace{1cm} & \hspace{1cm} & \hspace{1cm} \\
      \end{tabular}
      } % end of scope of "\resizebox"  directive
   \end{table}
   BUY order is totally executed $\rightarrow$ removed from order book. \\
   SELL order is partly executed $\rightarrow$ quantity reduced to {\bf 14}.
\end{frame}

\begin{frame}
\frametitle{Order book UML}
   \begin{minipage}{0.48\linewidth}
      First define the main objects of the model:
      \begin{itemize}
         \item A market contains books for instruments.
         \item Each book can contain orders.
      \end{itemize}
   \end{minipage}
   \begin{minipage}{0.48\linewidth}
      \begin{figure}
         \centering
         \includegraphics[height=0.9\textheight]{../images/order_book.pdf}
      \end{figure}
   \end{minipage}
\end{frame}

\begin{frame}
\frametitle{Order book UML}
   \begin{minipage}{0.48\linewidth}
      Then refine each object requirements:
      \begin{itemize}
         \item The market gives access to its books.
         \item Books have a name.
         \item User needs a method to insert orders in books.
         \item Orders of a book are defined by their quantity, price and side.
      \end{itemize}
   \end{minipage}
   \begin{minipage}{0.48\linewidth}
      \begin{figure}
         \centering
         \includegraphics[width = \textwidth]{../images/order_book2.pdf}
      \end{figure}
   \end{minipage}
\end{frame}

\begin{frame}
\frametitle{Order book UML}
   \begin{minipage}{0.48\linewidth}
      Finally think about technical implementation:
      \begin{itemize}
         \item Each side of the order book is an ordered list of orders.
         \item So orders on the same side can be compared.
         \item Need a priority to identify the first order at a price level.
      \end{itemize}
   \end{minipage}
   \begin{minipage}{0.48\linewidth}
      \begin{figure}
         \centering
         \includegraphics[width = \textwidth]{../images/order_book3.pdf}
      \end{figure}
   \end{minipage}
\end{frame}



\end{document}