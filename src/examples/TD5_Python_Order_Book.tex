\input{TD_class.tex}
\setcounter{section}{4}

\begin{document}

%----------------------------------------------------------------------------------------
\section{Market project}
%----------------------------------------------------------------------------------------

These exercices will be graded on 5 points.

Only working code pushed on a public github or gitlab repository will be examined.
The URL for your public repository will be asked during the MCQ exam.

\textbf{Please commit at least after each exercise, you will also be evaluated on your commit messages readability and usefulness.}

\textit{You are strongly advised to push to the remote repository after each exercise.}

%----------------------------------------------------------------------------------------
\subsection{Setup a basic python project}
%----------------------------------------------------------------------------------------

\begin{enumerate}
    \item Create a new git repository for this project.
    \item Switch to the 'main' branch.
    \item Create a virtualenv and add it to your .gitignore (it should not be committed).
    \item Export your dependencies using \verb+pip freeze > requirements.txt+ and make sure to commit that file.
        It can be used with \verb+pip install -r requirements.txt+ to reload the dependencies in a new virtualenv.
        You will need to rewrite it whenever you install new libraries in your project.
\end{enumerate}

%----------------------------------------------------------------------------------------
\subsection{Create the required objects}
%----------------------------------------------------------------------------------------

\begin{enumerate}
    \item Create the following file:
        \begin{verbatim}
        #!/usr/bin/env python3

        from book import Book

        def main():
            book = Book("TEST")
            book.insert_buy(10, 10.0)
            book.insert_sell(120, 12.0)
            book.insert_buy(5, 10.0)
            book.insert_buy(2, 11.0)
            book.insert_sell(1, 10.0)
            book.insert_sell(10, 10.0)

        if __name__ == "__main__":
            main()
        \end{verbatim}
    \item Create a book.py file with Book and Order objects.
\end{enumerate}


%----------------------------------------------------------------------------------------
\subsection{First implementation}
%----------------------------------------------------------------------------------------

Implement the Book and Order objects to obtain a result log showing for each insert call
\begin{itemize}
    \item the insert action details,
    \item the executed orders if any,
    \item the book state after insertion and execution.
\end{itemize}

Here is an example:
\begin{verbatim}
--- Insert BUY 10@10.0 id=1 on TEST
Book on TEST
        BUY 10@10.0 id=1
------------------------
--- Insert SELL 120@12.0 id=2 on TEST
Book on TEST
        SELL 120@12.0 id=2
        BUY 10@10.0 id=1
------------------------
--- Insert BUY 5@10.0 id=3 on TEST
Book on TEST
        SELL 120@12.0 id=2
        BUY 10@10.0 id=1
        BUY 5@10.0 id=3
------------------------
--- Insert BUY 2@11.0 id=4 on TEST
Book on TEST
        SELL 120@12.0 id=2
        BUY 2@11.0 id=4
        BUY 10@10.0 id=1
        BUY 5@10.0 id=3
------------------------
--- Insert SELL 1@10.0 id=5 on TEST
Execute 1 at 11.0 on TEST
Book on TEST
        SELL 120@12.0 id=2
        BUY 1@11.0 id=4
        BUY 10@10.0 id=1
        BUY 5@10.0 id=3
------------------------
--- Insert SELL 10@10.0 id=6 on TEST
Execute 1 at 11.0 on TEST
Execute 9 at 10.0 on TEST
Book on TEST
        SELL 120@12.0 id=2
        BUY 1@10.0 id=1
        BUY 5@10.0 id=3
------------------------
\end{verbatim}

\textit{The layout can be slightly different but the quantities and prices values must be the same.}

%----------------------------------------------------------------------------------------
\subsection{Write a bash script to launch the main.py class}
%----------------------------------------------------------------------------------------

Write a bash script named \textbf{run.sh} that

\begin{enumerate}
    \item Create the virtualenv if it does not exists, and load the dependencies from the requirements.txt.
    \item Activate the virtualenv if it exists.
    \item Run the main.py Python file.
\end{enumerate}

\textit{hint: \url{https://www.google.com/search?client=firefox-b-d&q=bash+how+to+test+file+exists}}

\textit{The corrector will use the bash script to check the python code behavior. If it does not work, he won't be able to validate it.}

%----------------------------------------------------------------------------------------
\subsection{Add pandas for visualization}
%----------------------------------------------------------------------------------------

\begin{enumerate}
    \item Create a new 'pandas' branch in your project.
    \item When building the string representation of your book
        \begin{itemize}
            \item create a panda DataFrame for each side
            \item fill it with the orders
            \item print it to have a tabular layout
        \end{itemize}
    \item Commit the changes.
    \item Merge this branch into the 'main' branch.    
\end{enumerate}

\end{document}
