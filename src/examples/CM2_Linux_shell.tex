\input{CM_class.tex}

\begin{document}

\begin{frame}
   \titlepage
\end{frame}

\begin{frame}
   \tableofcontents
\end{frame}

\section{Linux}

%----------------------------------------------------------------------------------------
\subsection{Linux shells}
%----------------------------------------------------------------------------------------

\begin{frame}
   \frametitle{History of shell}
   \begin{figure}
      \centering
      \includegraphics[width = \textwidth]{../images/shell-history.jpg}
   \end{figure}
\end{frame}

\begin{frame}
   \frametitle{Most common shells}
   \begin{itemize}
      \item {\bf sh} Bourne shell - 1979 \\
            Developed by Stephen Bourne. \\
            Minimal shell to expect on a *nix system.
      \item {\bf bash} Bourne-Again shell - 1989 \\
            GNU project to improve the Bourne shell. \\
            Default shell for all Linux distributions.
      \item {\bf zsh} Z shell - 1990 \\
            A more advanced shell, prettier and easier to customize. \\
            Maintains compatibility with the Bourne-Again shell.
      \item {\bf fish} Fish shell - 2005 \\
            "Finally, a command line shell for the 90s" \\
            Much easier and nicer to use. \\
            NOT compatible with bash scripts.
   \end{itemize}
\end{frame}

\begin{frame}
   \frametitle{Demo}
   \begin{figure}
      \centering
      \includegraphics[width = \textwidth]{../images/shell_demo.png}
   \end{figure}
\end{frame}

%----------------------------------------------------------------------------------------
\subsection{Basic shell commands}
%----------------------------------------------------------------------------------------

\begin{frame}
\frametitle{Explore filesystem}
   \begin{itemize}
      \item {\bf cd $<$path$>$} change directory \\
            Move working directory to the path.
      \item {\bf ls [-a] $<$path$>$} list \\
            List all (hidden) files in path.
      \item {\bf pwd} print working directory \\
            Display the path for working directory.
      \item {\bf which $<$command$>$} \\
            Displays the path for the command executable.
   \end{itemize}
\end{frame}

\begin{frame}[fragile]
   \frametitle{Example}
   \begin{minted}{bash}
   $ cd /home/galvarez/esilv/
   $ ls
   README.md   esilv.sh*   linux_os_HO.tex
   esilv.pdf   esilv.tex   linux_shell_GA.tex
   $ pwd
   /home/galvarez/esilv
   $ which pwd
   /usr/bin/pwd
   \end{minted}
\end{frame}

\begin{frame}
   \frametitle{Simple but important}
   \begin{itemize}
      \item \textbf{Q} must be pressed once to quit messages like a command manual.
      \item \textbf{yes/no}, \textbf{y/n}, \textbf{Y/n}. Write one of them and press Enter. It is used to confirm/refuse an action (e.g. when installing a package).
      \item Password letters are hidden. You must type without seeing what you are doing (e.g. when connecting to a server, when using sudo).
   \end{itemize}
\end{frame}

\begin{frame}
   \frametitle{Edit a (text) file}
   You can use any editor of your choice. Often pre-installed editors are Vim or Nano. Basic use of Vim:
   \begin{itemize}
      \item 2 modes: visual mode, edition mode
      \item Enter edition mode: \textbf{i}, or \textbf{a}, etc.
      \item Quit edition mode: \textbf{ESC}
      \item Quit Vim! In visual mode, press \textbf{:}, then:
      \begin{itemize}
         \item \textbf{wq} and press Enter, for saving and quiting.
         \item \textbf{q!} and press Enter, for quiting without saving. Useful when you entered a read-only file.
      \end{itemize}
      \item Level up in Vim to copy, paste, jump to rows, specific words, etc.
   \end{itemize}
\end{frame}

\begin{frame}[fragile]
   \frametitle{Getting help}
   You can get information on the use of a command and its parameters
   \begin{itemize}
      \item {\bf <command> --help}
      \item {\bf man <command>}
   \end{itemize}
\end{frame}

\begin{frame}[fragile]
   \frametitle{Setup an environment}
      \begin{itemize}
         \item {\bf sudo $<$command$>$} substitute user do \\
               Execute administrator actions, require a password.
         \item {\bf apt $<$install$>$ $<$package$>$} Advanced Packaging Tool \\
               Install and update packages.
         \item {\bf source} \\
               Execute a script, similar to
               \begin{minted}{bash}
                  $ ./myscript.sh
                  $ . myscript.sh
               \end{minted}
      \end{itemize}
   \end{frame}

\begin{frame}[fragile]
   \frametitle{Example - apt install}
   \begin{minted}[fontsize=\small]{bash}
   $ sudo apt install postgresql-contrib
   Reading package lists... Done
   Building dependency tree
   Reading state information... Done
   postgresql-contrib is already the newest version (12+214ubuntu0.1).
   0 upgraded, 0 newly installed, 0 to remove and 13 not upgraded.
   \end{minted}
\end{frame}

\begin{frame}[fragile]
   \frametitle{Example - apt remove}
   \begin{minted}[fontsize=\small]{bash}
   $ sudo apt --purge remove postgresql-contrib
   Reading package lists... Done
   Building dependency tree
   Reading state information... Done
   The following packages will be REMOVED:
     postgresql-contrib*
   0 upgraded, 0 newly installed, 1 to remove and 13 not upgraded.
   After this operation, 67.6 kB disk space will be freed.
   Do you want to continue? [Y/n]
   \end{minted}
\end{frame}

\begin{frame}[fragile]
   \frametitle{System parameters - PATH}
   PATH
   \begin{itemize}
      \item Variable giving the paths to launch applications
      \item Get them using:
      \begin{minted}[fontsize=\small]{bash}
         $ echo $PATH
         /usr/local/sbin:/usr/local/bin:/usr/sbin:/usr/bin:/sbin:/bin:/usr/games:/usr/local/games:/snap/bin:
      \end{minted}
      \item Temporarilly add an application path (for the session):
      \begin{minted}[fontsize=\small]{bash}
            $ export PATH=$PATH:/home/user/path_to_app
      \end{minted}
      \item Permanently add an application path (require relaunch):
      \begin{minted}[fontsize=\small]{bash}
            $ echo 'export PATH=$PATH:/path_to_app' >> /home/user/.profile
      \end{minted}
   \end{itemize}
\end{frame}

\begin{frame}[fragile]
   \frametitle{System parameters - .profile}
   These files contain system and user parameters: .profile or .bashrc, or other
   \begin{minted}[fontsize=\small]{bash}
      $ cat /home/user/.profile
      # ~/.profile: executed by the command interpreter for login shells.
      # This file is [...]

      # set PATH so it includes user's private bin if it exists
      if [ -d "$HOME/bin" ] ; then
          PATH="$HOME/bin:$PATH"
      fi
   \end{minted}
\end{frame}

\begin{frame}[fragile]
   \frametitle{System parameters - .profile}
   Explore some of the locations (these might slightly differ from your configuration)
   \begin{minted}[fontsize=\small]{bash}
      $ cd /usr/local/sbin
      $ cd /usr/local/bin
      $ cd /usr/sbin
      $ cd /usr/bin
      $ ls | grep ssh
      $ ls | grep scp
      $ ls | grep grep
      $ ls | grep ls
   \end{minted}
\end{frame}

\begin{frame}[fragile]
   \frametitle{Control structures}
   \begin{itemize}
      \item {\bf pipeline} command1 $|$ command2 \\
            Pass a command output as input for the next one.
      \item {\bf output redirection} command $>$ file \\
            Write a command output into a file. \\
            command $>>$ file \\
            Append a command output to a file. \\
            command $<$ file \\
            Use a file as a command input.
      \item {\bf pathname expansion a.k.a globbing} command *json \\
            Command targets all files ending with json.
      \item {\bf command substitution} command2 $`$command1$`$ \\
            Use a command input as parameter for another command.
      \item {\bf for} for i in 'some list' ; do echo "\$i" ; done \\
            Execute a command on every item of a list of entries.
   \end{itemize}
\end{frame}

\begin{frame}[fragile]
   \frametitle{Example - for loop}
   Add content of many files in a single file
   \begin{minted}[fontsize=\small]{bash}
      for f in $(ls *.txt)
      do
          cat $f >> testdir/big_file.txt
      done
   \end{minted}
\end{frame}


\begin{frame}[fragile]
   \frametitle{Work with a remote instance (server)}

   Connect using SSH to the remote instance and use it as your own
   \begin{itemize}
      \item With \textit{username} the username used when using the remote server.
      \item Example
      \begin{minted}[fontsize=\small]{bash}
         $ # ssh -i [path_private_key] [username]@[remote_server_address]
         $ ssh -i ~/.pssd/linux_course_key.pem ubuntu@ec2-52-14-202-75.us-east-2.compute.amazonaws.com
      \end{minted}
   \end{itemize}
\end{frame}


\begin{frame}[fragile]
   \frametitle{Work with a remote instance (server)}

   Exchange files from local to remote server using \textbf{SCP}
   \begin{itemize}
      \item Example
      \begin{minted}[fontsize=\small]{bash}
         $ # scp -i [path_private_key] [path_local_file] [username]@[remote_server_address]:[path_remote]
         $ scp -i ~/.pssd/linux_course_key.pem dumb.txt ubuntu@ec2-52-14-202-75.us-east-2.compute.amazonaws.com:~/
      \end{minted}
   \end{itemize}
\end{frame}

\begin{frame}[fragile]
   \frametitle{Work with a remote instance (server)}
   Exchange files from remote server to local using \textbf{SCP}
   \begin{itemize}
      \item Example
      \begin{minted}[fontsize=\small]{bash}
         $ # scp -i [path_private_key] [username]@[remote_server_address]:[path_remote_file] [path_local]
         $ scp -i ~/.pssd/linux_course_key.pem ubuntu@ec2-52-14-202-75.us-east-2.compute.amazonaws.com:dumby.txt .
      \end{minted}
   \end{itemize}
   Careful: On Mac, don't forget to set the permissions to read for the user only. chmod 400 private\_key.pem
   Otherwise it will be blocked.
\end{frame}


\begin{frame}
   \frametitle{Manipulate data}
   \begin{itemize}
      \item {\bf wget} World Wide Web GET \\
            Download a file from an URL.
      \item {\bf curl} Client URL \\
            Execute any HTTP command (GET by default), print received data.
      \item {\bf grep} Global Regular Expressions Print \\
            Select parts of files that match a pattern.
      \item {\bf sed} Stream EDitor \\
            Parses and transform text data.
   \end{itemize}
\end{frame}

\begin{frame}[fragile]
   \frametitle{Example - wget}
   \begin{minted}[fontsize=\small]{bash}
   $ wget perdu.com
   --2020-12-01 18:19:01--  http://perdu.com/
   Resolving perdu.com (perdu.com)... 208.97.177.124
   Connecting to perdu.com (perdu.com)|208.97.177.124|:80... connected.
   HTTP request sent, awaiting response... 200 OK
   Length: 204 [text/html]
   Saving to: ‘index.html’

   index.html  100%[==========>]     204  --.-KB/s    in 0s

   2020-12-01 18:19:01 (45.2 MB/s) - ‘index.html’ saved [204/204]
   $ head index.html
   <html><head><title>Vous Etes Perdu ?</title>...
   \end{minted}
\end{frame}

% # Wget
% wget perdu.com
% wget https://cdn.kernel.org/pub/linux/kernel/v4.x/linux-4.17.2.tar.xz


\begin{frame}[fragile]
   \frametitle{Example - curl}
   \begin{minted}{bash}
   $ curl -s perdu.com
   <html><head><title>Vous Etes Perdu ?</title>...
   \end{minted}
\end{frame}
% # CURL
% Ex1:
% curl -s https://www.google.fr/ | grep -E chance
% Ex2:
% curl https://en.wikipedia.org/wiki/Unix_philosophy

\begin{frame}[fragile]\frametitle{Example - grep}
   \small
   \begin{minted}{bash}
   $ ls | grep '.sh'
   esilv.sh
   linux_shell_GA.tex
   \end{minted}

   \begin{minted}{bash}
   $ ls | grep '\.sh'
   esilv.sh
   \end{minted}

   \begin{minted}{bash}
   $ ls | grep -o '.sh'
   .sh
   _sh
   \end{minted}

   \begin{minted}{bash}
   $ ls | grep -P 'esilv.[^p]]|sh'
   esilv.sh
   esilv.tex
   linux_shell_GA.tex
   \end{minted}

\end{frame}

\begin{frame}[fragile]
   \frametitle{Example - sed}
   \begin{minted}[fontsize=\small]{bash}
   $ echo 'Too" "ma"ny "double quotes"'
      | sed 's/"//g'
   Too many double quotes

   $ echo 'Too" "ma"ny "double quotes"'
      | sed -E 's/"([a-z])/\1"/g'
   Too" m"an"y d"ouble quotes"
   \end{minted}
\end{frame}

%----------------------------------------------------------------------------------------
\subsection{Use of Regular Expressions}
%----------------------------------------------------------------------------------------

\begin{frame}[fragile]
   \frametitle{Regular Expressions}
   Character sequence defining a character pattern.
   \begin{itemize}
      \item Match a text against a pattern
      \item Find a pattern in a text
      \item Substitute a pattern in a text
   \end{itemize}
\end{frame}

\begin{frame}[fragile]
   \frametitle{Regular Expressions - Examples}
   Character sequence defining a search pattern.
   \begin{itemize}
      \item Recognize a character sequence \\
            "aabcda" + r"(bc)" $\rightarrow$ "bc"
      \item With alternatives \\
            "aabcda" + r"(b$|$c)" $\rightarrow$ "b", "c" \\
            "aabcda" + r"([bcd])" $\rightarrow$ "b", "c", "d"
      \item And quantifiers \\
            "aabcda" + r"(a*b*)" $\rightarrow$ "aab", "a" \\
            "aabcda" + r"(a*b+)" $\rightarrow$ "aab"
      \item Greedy by default but can be lazy \\
            "aabcda" + r"(a*?)" $\rightarrow$ "a", "a", "a"
   \end{itemize}
\end{frame}

\begin{frame}[fragile]
   \frametitle{Regular Expressions - Useful links}
   Because syntax can quickly become very complex and it is easier to test within a sandbox:
   \begin{itemize}
      \item \url{https://regexr.com/}
      \item \url{https://www.regextester.com/}
      \item \url{https://regex101.com/}
   \end{itemize}
\end{frame}

%----------------------------------------------------------------------------------------
\subsection{Shell scripts}
%----------------------------------------------------------------------------------------

\begin{frame}[fragile]
   \frametitle{Shells provide a history}
   You will often execute the same commands many times.
   \begin{itemize}
      \item {\bf history} opens the complete history
      \item $\uparrow$ and $\downarrow$ allow to navigate in previous commands
      \item {\bf ctrl+R} allows to search among previous commands
      \item {\bf !!} executes the previous command
      \item recent shells (like Fish or ZSH) provide auto-completion
      \item and many other shell-specific tools...
   \end{itemize}
\end{frame}

\begin{frame}[fragile]
   \frametitle{But it is often not enough}
   Very easy to make mistakes
   \begin{itemize}
      \item most commands are very small, a typo leads to another command
      \item the shell is sensitive to case and spacing
      \item chaining commands compose their complexity
   \end{itemize}
\end{frame}

\begin{frame}[fragile]
   \frametitle{But it is often not enough}
   Often need to replicate the commands
   \begin{itemize}
      \item multiple times until they work
      \item on multiple files
      \item on multiple machines
   \end{itemize}
\end{frame}

\begin{frame}[fragile]\frametitle{Write commands in a script}
   \footnotesize
   Write in script.sh:
   \begin{minted}{bash}
      #!/bin/bash
      echo "Hello World, your files are"
      ls
   \end{minted}
   Make this file executable:
   \begin{minted}{bash}
      $ chmod u+x script.sh
   \end{minted}
   Then execute it:
   \begin{minted}{bash}
      $ ./script.sh
      Hello World, your files are
      README.md   esilv.sh*   linux_os_HO.tex
      $ . script.sh
      Hello World, your files are
      README.md   esilv.sh*   linux_os_HO.tex
      $ source script.sh
      Hello World, your files are
      README.md   esilv.sh*   linux_os_HO.tex
   \end{minted}
\end{frame}

\begin{frame}[fragile]
   \frametitle{Bash scripts can be dangerous}
   This is a valid bash script, will use all your memory and crash your machine:
   \begin{verbatim}
      :(){ :|:& };:
   \end{verbatim}
   It is called a \href{https://en.wikipedia.org/wiki/Fork_bomb}{Fork Bomb}.
\end{frame}

\begin{frame}[fragile]
   \frametitle{Bash scripts can be VERY dangerous}
   This is a valid bash script, will erase your hard drive:
   \begin{minted}{bash}
      $ sudo rm -rf /
   \end{minted}
\end{frame}

\begin{frame}[fragile]
   \frametitle{Bash is Turing-complete}
   So you can use it to solve any problem...
   \begin{minted}{bash}
      #!/bin/bash
      min() {
         printf "%s\n" "${@:2}" | sort "$1" | head -n1
      }
      max() {
         min ${1}r ${@:2}
      }
   \end{minted}
   ...but prefer using higher-level languages!
\end{frame}

%----------------------------------------------------------------------------------------
\section{Git}
\section{Python}

\end{document}
