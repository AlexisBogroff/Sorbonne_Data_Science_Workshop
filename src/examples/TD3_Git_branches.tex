\input{TD_class.tex}
\setcounter{section}{2}

\begin{document}

%----------------------------------------------------------------------------------------
\section{Git Branches}
%----------------------------------------------------------------------------------------

These exercises must be done by teams of 3-4 students.

%----------------------------------------------------------------------------------------
\subsection{Clone a Git repository}  % 5 to 10 min
%----------------------------------------------------------------------------------------

\begin{enumerate}
    \item Choose the repository created on GitHub or GitLab by one of your teammates, share its web URL.
    \item \textbf{For the repository owner...} Ensure there is at least a README.md file, it appears on the project frontpage in the web UI.
    \item Using only command-line in your Linux shell, clone it to a local repository.
    \item \textbf{For the repository owner...} Give push rights to your teammates:
        \begin{itemize}
            \item on \textit{GitHub} got to "Settings", "Manage access", "Invite a collaborator" \\
                  see https://docs.github.com/en/github/setting-up-and-managing-your-github-user-account/inviting-collaborators-to-a-personal-repository
            \item on \textit{GitLab} got to "Members", "Invite member" \\
                  see https://docs.gitlab.com/ee/user/project/members/
        \end{itemize}
\end{enumerate}

\textit{hint: All Git commands have a \textbf{-h} flag to display the corresponding help.}

\textit{hint: Unless you setup a SSH private key, you want to clone using the HTTPS address, not the SSH or CLI one.}

% git clone <url>

%----------------------------------------------------------------------------------------
\subsection{Push files to common repository}  % 10 to 20 min
%----------------------------------------------------------------------------------------

Using only the shell in your local repository:

\begin{enumerate}
    \item Create a branch named after you.
    \item Create a new text file named after you (with the content you want).
    \item Commit this new file.
    \item Push your branch to the remote repository.
\end{enumerate}

Check in the web UI you see your branch (there is a button with 'master' as default).
Ensure it contains your file with the same content you entered locally.

% git branch galvarez
% git switch galvarez
% echo "test" > galvarez.txt
% git add galvarez.txt
% git commit -m "New file from Guillaume Alvarez"
% git push origin galvarez

\textit{hint: Lost in your commits?} use \verb+git log+ or \verb+git log --graph --oneline+ to print the commit tree.

%----------------------------------------------------------------------------------------
\subsection{Merge simple changes}  % 10 to 20 min
%----------------------------------------------------------------------------------------

Using only the shell:
\begin{enumerate}
    \item Merge your branch into the 'master' branch.
    \item Push your changes in the 'master' branch to the remote repository.
\end{enumerate}

\textit{hint: You may have to merge or rebase on the changes from your teammates.}

Check in the web UI you see your own file in the 'master' branch.
Ensure it contains the same content you entered locally.

% git switch master
% git merge galvarez
% git push origin master

% may need rebase after other people in the team committed they own files
% git pull --rebase


%----------------------------------------------------------------------------------------
\subsection{Resolve merge conflicts}  % 20 to 40 min
%----------------------------------------------------------------------------------------

\textbf{For the repository owner...} In the web UI, make sure lines 2 to 6 of the README.md are not empty.

\textit{One person at a time}, using only the shell:
\begin{enumerate}
    \item Switch back to your own branch (not including the latest changes from the master branch).
    \item Edit the lines 2 to 6 of the README.md file with a text you like (a poem, a quote, some clever code...).
          It can be any readable text, it may be incomplete, it must just take about 5 lines and be different from your teammates.
          It must start on line 2 to trigger conflicts between team members.
    \item Commit this change.
    \item Pull latest status from the remote repository 'master' branch into your local 'master' branch.
    \item Merge your branch into the local 'master' branch.
    \item If there are conflicts, \textit{we want the paragraph to appear in alphabetical order in the final README.md file.}
    \item Push your changes in the 'master' branch to the remote repository.
\end{enumerate}

\textit{hint: You may edit the README.md file using nano or vim in shell, nano may be easier as it displays available commands at the bottom.}

Check the README.md content in the web UI.
After everyone in the group made their change and resolved conflicts, the file in 'master' branch should contain one paragraph per team member.

% git switch galvarez
% vim README.md
% git add README.md
% git commit -m "Add a paragraph"
% git switch master 
% git pull --rebase
% git merge galvarez
% vim README.md to resolve the conflict
% git add README.md
% git commit -m "Add a paragraph"
% git push origin master

%----------------------------------------------------------------------------------------
\subsection{Take latest changes from master in local branch}  % 10 to 20 min
%----------------------------------------------------------------------------------------

\textbf{For the repository owner...} In the web UI, add a line of text at the beginning of the README.md with the team members' names or aliases.

Using only the shell in your local repository:
\begin{enumerate}
    \item Pull the latest changes in the 'master' branch, check the README.md is up-to-date (contains all the paragraphs and the new line).
    \item Switch back to your own branch (not including the latest changes from the master branch).
    \item Merge the changes from 'master' to your own branch.
    \item Commit this change.
\end{enumerate}

Check the README.md content in the web UI.
After everyone in the group made their change and resolved conflicts, the file in 'master' branch should contain one paragraph per team member.

% git pull --rebase
% git switch galvarez
% git merge master
% vim README.md to resolve the conflict
% git add README.md
% git commit -m "Merge latest changes from master"

%----------------------------------------------------------------------------------------
\subsection{Delete a branch}  % 5 to 10 min
%----------------------------------------------------------------------------------------

Using only the shell in your local repository:
\begin{enumerate}
    \item Delete your branch on local repository.
    \item Delete your branch on distant repository.
\end{enumerate}

Using the web UI, ensure only the 'master' branch remains.

%----------------------------------------------------------------------------------------
\subsection{Rebase interactively to have a clean history}  % 20 to 30 min
%----------------------------------------------------------------------------------------

Using only the shell in your local repository:
\begin{enumerate}
    \item Pull the latest changes in the 'master' branch.
    \item Create a new local branch named after you and switch to it.
    \item Then \textbf{with a separate commit for each change}:
        \begin{enumerate}
            \item Clear the whole file, removing all text.
            \item Add a title line "Git interactive rebase".
            \item Copy the first paragraph from https://git-scm.com/book/en/v2/Git-Tools-Rewriting-History.
            \item Add the second paragraph from the same page.
            \item Add the first and second paragraphs from the \textit{"Changing Multiple Commit Messages"} section in the same page.
            \item Remove the second paragraph from your file.
            \item Add the missing title \textit{"Changing Multiple Commit Messages"} on a line just before the two paragraphs your copied (before \textit{To modify a commit that is farther back in your history...}).
            \item Add a final line with your name or alias.
        \end{enumerate}
        The commit history of your branch should then be a bit messy with 8 commits.
    \item Use interactive rebase to have a single commit with message \textit{"Explain git interactive rebase."}.
    \item Push your branch on the remote repository.
\end{enumerate}

\textit{hint: You can change Git default editor with command \textbf{git config --global core.editor "path to editor"}.
By default it uses \textbf{nano}.}

Using the web UI, check the README.md content in your branch on the remote repository.
Check the your branch history or graph, it should only contain the aggregated commit you pushed, not all the local commits.
The 'master' branch should not have changed.

%----------------------------------------------------------------------------------------
\subsection{Create and approve a Merge/Pull Request}  % 20 to 30 min
%----------------------------------------------------------------------------------------

In the web UI open your branch then
\begin{itemize}
    \item In \textbf{GitLab} click on \textit{'Create merge request'} to create a Merge Request to merge your branch into the 'master' branch.
    \item In \textbf{GitHub} click on \textit{'pull request'} to create a Pull Request to merge your branch into the 'master' branch.
    \item Ask another team member to check there is a single commit and merge the Merge/Pull Request.
\end{itemize}

\textit{hint: As multiple people will try to change the same file, you may well have conflicts.
In that case you have to rebase your branch on the latest state of the 'master' branch, resolving potential conflicts, and push it again.
Reloading the merge/pull request webpage will update it.}


\end{document}
