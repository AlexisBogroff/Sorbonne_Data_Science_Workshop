\documentclass{beamer}
\usetheme{Warsaw}
\setbeamertemplate{headline}{}

\usepackage{ae,lmodern}
\usepackage[english]{babel}
\usepackage[utf8]{inputenc}
\usepackage[T1]{fontenc}

\usepackage{caption}
\captionsetup[figure]{labelformat=empty}

\PassOptionsToPackage{usenames,dvipsnames}{xcolor}
\usepackage{xcolor,colortbl}
\definecolor{DarkGrey}{HTML}{222222}
\definecolor{DarkBlue}{HTML}{004BA9}
\definecolor{DarkRed}{HTML}{CC1111}
\definecolor{DarkGreen}{HTML}{117711}
\definecolor{DarkOrange}{HTML}{CC7000}
\definecolor{LightGrey}{HTML}{DDDDDD}
\definecolor{LightBlue}{HTML}{F0F8FF}
\definecolor{codegreen}{rgb}{0,0.6,0}
\definecolor{codepurple}{rgb}{0.58,0,0.82}

\usepackage[cache=false]{minted}
\setminted[bash]{
   bgcolor=LightBlue,
   breaklines, breakanywhere,
   frame=single,
   autogobble
}
\usemintedstyle[python]{native}
\setminted[python]{
   bgcolor=black,
   breaklines, breakanywhere,
   autogobble
}

\usepackage{listings}
\usepackage{lstautogobble}
\lstdefinestyle{bash}{
    backgroundcolor=\color{DarkGrey},   
    commentstyle=\color{codegreen},
    keywordstyle=\color{magenta},
    numberstyle=\tiny\color{DarkGrey},
    stringstyle=\color{codepurple},
    basicstyle=\ttfamily\tiny\color{LightGrey},
    escapeinside={\%*}{*)},
    breakatwhitespace=false,         
    breaklines=true,                 
    captionpos=b,                    
    keepspaces=true,                 
    numbers=left,                    
    numbersep=5pt,                  
    showspaces=false,                
    showstringspaces=false,
    showtabs=false,
    showlines=false,
    tabsize=2
}

\usepackage{tikz}
\usetikzlibrary{calc,decorations.pathreplacing,arrows,arrows.meta,shapes,patterns, positioning}
\newcommand\BigLength{14.6em}
\newcommand\Height{2em}
\newcommand\Sep{0.6em}
\newcommand\Center{\BigLength*1/2}
\newcommand\BigBox{\BigLength+\Sep}
\newcommand\HalfBox{\BigLength*1/2-\Sep*1/4}
\newcommand\HalfLength{\BigLength*1/2-\Sep*5/4}
\newcommand\CenterL{\BigLength*1/4-\Sep*1/8}
\newcommand\CenterR{\BigLength*3/4+\Sep*1/8}
\tikzstyle{layer}=[rectangle,thick,text centered,
                     minimum height=\Height,minimum width=\BigLength]
\tikzstyle{short}=[rectangle,thick,text centered,
                     minimum height=\Height,minimum width=\HalfLength]
\tikzstyle{dibox}=[rectangle,thick,semitransparent,
                     minimum height=(\Height+\Sep)*2,minimum width=\BigBox]
\tikzstyle{vmbox}=[rectangle,thick,semitransparent,
                     minimum height=(\Height+\Sep)*3,minimum width=\HalfBox]
\tikzstyle{ctbox}=[rectangle,thick,semitransparent,
                     minimum height=(\Height+\Sep)*2,minimum width=\HalfBox]
\tikzstyle{vebox}=[rectangle,thick,semitransparent,
                     minimum height=(\Height+\Sep)*1,minimum width=\HalfBox]

\usepackage{hyperref}
\usepackage{grffile}


\AtBeginSection[]
{
   \begin{frame}
      \tableofcontents[currentsection]
   \end{frame}
}

\AtBeginSubsection[]
{
   \begin{frame}
      \tableofcontents[currentsection, currentsubsection, sectionstyle=shaded]
   \end{frame}
}

%----------------------------------------------------------------------------------------
\title{Introduction to Data Science}
\subtitle{with Python}
%----------------------------------------------------------------------------------------
\author{Alexis Bogroff}
\date{\today}



\begin{document}

\begin{frame}
   \titlepage
\end{frame}

\begin{frame}
   \tableofcontents
\end{frame}

\section{Linux}
\section{Git}

%----------------------------------------------------------------------------------------
\subsection{Git distant repositories}
%----------------------------------------------------------------------------------------

% for pretty graphs:
% http://jsfiddle.net/4t2cpgjo/57/
% can save the image in the bottom right corner

\begin{frame}
\frametitle{Local Git repository}
   Contains Git objects as explained in previous lesson.
   \begin{itemize}
      \item Stores a history of changes (a.k.a. commits).
      \item Allows to revert back to any state of the history.
   \end{itemize}
\end{frame}

\begin{frame}
\frametitle{Local Git repository - some commands}
   \begin{itemize}
      \item {\bf git init} Create a Git repository from an existing folder.
      \item {\bf git add} Register some changes for next commit.
      \item {\bf git commit} Add changes to local Git history.
      \item {\bf git checkout} Change files to another point in Git history.
      \item ...
   \end{itemize}
\end{frame}

\begin{frame}
\frametitle{Distant Git repository}
   Same as a local Git repository...
   \begin{itemize}
      \item Stores a history of changes (a.k.a. commits).
      \item Allows to revert back to any state of the history.
   \end{itemize}
   ...on a distant server.
   \begin{itemize}
      \item Can send and receive commits and other Git objects.
   \end{itemize}
\end{frame}

\begin{frame}
\frametitle{Distant Git repository - new commands}
   \begin{itemize}
      \item {\bf git clone} Copy a distant repository.
      \item {\bf git fetch} Download Git objects from distant repository.
      \item {\bf git merge} Merge two sets of {\it local} Git objects, by default the local and remote (i.e. already downloaded from distant repository) ones.
      \item {\bf git pull} Fetch AND merge from distant repository.
      \item {\bf git push} Send Git objects to a distant repository.
   \end{itemize}
\end{frame}

\begin{frame}[fragile]
\frametitle{Edit a distant repository: clone/commit/push}
   \small
   When starting to work on a repository:
   \begin{minted}{bash}
      $ git clone https://gitlab.hsoftware.com/galvarez/test-project.git
      $ cd test-project
      $ echo "some changes" >> README.md
      $ git add README.md
      $ git commit -m "Some changes in readme"
      $ git push
   \end{minted}
\end{frame}

\begin{frame}[fragile]
\frametitle{Edit a distant repository: pull/commit/push}
   When resuming work on a repository:
   \begin{minted}{bash}
      $ git pull --rebase
      $ echo "some more changes" >> README.md
      $ git add README.md
      $ git commit -m "More changes in readme"
      $ git push
   \end{minted}
\end{frame}

\begin{frame}[fragile]
\frametitle{Note on git push}
   With default naming, these are equivalent:
   \begin{minted}{bash}
      git push
      git push origin master
   \end{minted}

   They both mean
   \begin{quotation}
      push to the 'master' branch on 'origin' repository
   \end{quotation}
   'origin' is the usual default name for a distant repository.
   'master' is the default branch (it may change due to the \href{https://tools.ietf.org/id/draft-knodel-terminology-00.html#rfc.section.1.1}{master-slave} issue).
\end{frame}

\begin{frame}
\frametitle{Edit a distant repository}
   Demo...
   \begin{figure}
      \centering
      \includegraphics[width = \textwidth]{../images/demo_edit_distant_repo.png}
   \end{figure}
\end{frame}

\begin{frame}
\frametitle{Cloud repository managers}
   Provide support for basic Git commands and devops features likes
   issue tracking, wiki, continuous integration, deployment pipelines, etc.
   \begin{itemize}
      \item https://github.com/ (cloud only)
      \item https://gitlab.com/ (cloud + on-premise)
      \item https://bitbucket.org/ (cloud + on-premise)
   \end{itemize}
\end{frame}

%----------------------------------------------------------------------------------------
\subsection{Git tags}
%----------------------------------------------------------------------------------------

\begin{frame}
\frametitle{Tags}
   Tags mark a commit.
   \begin{itemize}
      \item Provide information on that commit.
      \item Shortcut to that commit.
   \end{itemize}
\end{frame}

\begin{frame}\frametitle{Tags - commands}
   \small
   The main command is {\bf git tag}:
   \mint{bash}|git tag|
   List all tags in local repository.
   \mint{bash}|git tag -l "v1*"|
   List tags starting with 'v1' in local repository.
   \mint{bash}|git tag v1.2|
   Tag last commit with tag 'v1.2' in local repository.
   \mint{bash}|git tag -d v1.2|
   Delete tag 'v1.2' from local repository. \\
   The commit is NOT deleted!
\end{frame}

\begin{frame}[fragile]
\frametitle{Tags - commands}
   You can also roll back your history to a past tag:
   \mint{bash}|git checkout v1.2|
   Check out tag 'v1.2' from local repository \\
   It changes your local files to their state in tag v1.2. \\
   \alert{With a huge warning in shell:}
   \begin{minted}[fontsize=\tiny]{bash}
   Note: switching to 'v1.2'.

   You are in 'detached HEAD' state. You can look around, make experimental
   changes and commit them, and you can discard any commits you make in this
   state without impacting any branches by switching back to a branch.

   If you want to create a new branch to retain commits you create, you may
   do so (now or later) by using -c with the switch command. Example:

      git switch -c <new-branch-name>

   Or undo this operation with:

      git switch -

   Turn off this advice by setting config variable advice.detachedHead to false
   \end{minted}
\end{frame}

\begin{frame}[fragile]
\frametitle{Tags - commands}
   \footnotesize
   \alert{Warning:} \\
   The {\bf git push}, {\bf git fetch} and {\bf git pull} commands do not push/fetch/pull tags by default! Thus, use one of:\\
   Push all tags to distant repository.
   \mint{bash}|git push --tags|
   Push tag 'v1.2' to distant repository.
   \mint{bash}|git push origin v1.2|
   Delete tag 'v1.2' from distant repository.
   \mint{bash}|git push origin -d v1.2|
   Download all tags from distant repository.
   \begin{minted}{bash}
      git fetch --tags
      git pull --tags
   \end{minted}
\end{frame}

%----------------------------------------------------------------------------------------
\subsection{Git branches}
%----------------------------------------------------------------------------------------

\begin{frame}
\frametitle{Why use branches?}
   \begin{itemize}
      \item<1-> Try out new ideas.
      \item<2-> Work on your own state and still save on common server.
      \item<3-> Prepare changes to the common code.
      \item<4-> Separate a specific set of changes.
   \end{itemize}
\end{frame}

\begin{frame}
\frametitle{Why use branches?}
   \begin{figure}
      \centering
      \includegraphics[width = \textwidth]{../images/graphs/branches.png}
   \end{figure}
\end{frame}

\begin{frame}
\frametitle{What are branches?}
   Git is an oriented tree of commits.
   \begin{itemize}
      \item The leafs are the HEAD commits, those without children.
      \item A 'branch' is a reference to a head (= leaf) commit and all its parents.
      \item By convention a branch designates the referenced commit up to its ancestor.
   \end{itemize}
\end{frame}

\begin{frame}
\frametitle{What are branches?}
   \begin{figure}
      \centering
      \includegraphics[width = \textwidth]{../images/graphs/branches.png}
   \end{figure}
\end{frame}

\begin{frame}
\frametitle{What are branches?}
   \begin{figure}
      \centering
      \includegraphics[width = \textwidth]{../images/graphs/branches_with_merge.png}
   \end{figure}
\end{frame}

\begin{frame}
\frametitle{What are branches?}
   \begin{figure}
      \centering
      \includegraphics[width = \textwidth]{../images/graphs/branches_with_merge2.png}
   \end{figure}
\end{frame}

\begin{frame}[fragile]
\frametitle{Branches - commands}
   \scriptsize
   The main command is {\bf git branch}:
   List all branches in local repository.
   \mint{bash}|git branch|
   List branches starting with 'v1' in local repository.
   \mint{bash}|git branch -l "v1*"|
   Start branch 'v1.2' from last commit with tag  in local repository.
   \mint{bash}|git branch v1.2|
   Delete branch 'v1.2' from local repository. \\
   \mint{bash}|git branch -d v1.2|
   It will gently warn you though:
   \alert{The commits ARE deleted!}
   It is very similar to {\bf git tag}.
   \begin{minted}[fontsize=\tiny]{bash}
      $ git branch -d v1.2
      error: The branch 'v1.2' is not fully merged.
      If you are sure you want to delete it, run 'git branch -D v1.2'.
   \end{minted}
\end{frame}

\begin{frame}[fragile]
\frametitle{Branches - commands}
   Push the current branch to default distant repository.
   \mint{bash}|git push|
   Push branch 'v1.2' to distant repository.
   \mint{bash}|git push origin v1.2|
   Delete branch 'v1.2' from distant repository.
   \mint{bash}|git push origin -d v1.2|
   Download (and merge) branch 'v1.2' from distant repository.
   \begin{minted}{bash}
      git fetch origin v1.2
      git pull origin v1.2
   \end{minted}
\end{frame}

\begin{frame}
\frametitle{How to switch branches?}
   There are different methods:\\
   Load files from branch 'v1.2':
   \mint{bash}|git checkout v1.2|
   Or:
   \mint{bash}|git switch v1.2|
   Create branch 'v1.2' and start working on it:
   \mint{bash}|git checkout -b v1.2|
\end{frame}

\begin{frame}
\frametitle{How to merge branches?}
   The main command is {\bf git merge}:\\
   Merge commits from branch 'v1.2' with current branch:
   \mint{bash}|git merge v1.2|
   It creates a merge commit with two parents
   \begin{minipage}{0.48\linewidth}
      \begin{figure}
         \centering
         \includegraphics[width = \textwidth]{../images/graphs/branches_merge_before.png}
      \end{figure}
   \end{minipage}
   \begin{minipage}{0.48\linewidth}
      \begin{figure}
         \centering
         \includegraphics[width = \textwidth]{../images/graphs/branches_merge_after.png}
      \end{figure}
   \end{minipage}
\end{frame}

\begin{frame}
\frametitle{How to merge branches?}
   Another solution for {\it fast-forward changes} is {\bf git rebase}:
   Rebase commits from branch 'v1.2' after master branch:
   \mint{bash}|git rebase master v1.2|
   It does NOT create a merge commit.
   \begin{minipage}{0.48\linewidth}
      \begin{figure}
         \centering
         \includegraphics[width = \textwidth]{../images/graphs/branches_merge_before.png}
      \end{figure}
   \end{minipage}
   \begin{minipage}{0.48\linewidth}
      \begin{figure}
         \centering
         \includegraphics[width = \textwidth]{../images/graphs/branches_rebase_after.png}
      \end{figure}
   \end{minipage}
\end{frame}

\begin{frame}
\frametitle{Warning on git rebase}
   \alert{It changes the commit history:}
   \begin{itemize}
      \item The commit id (SHA-1) depends on its parent and content.
      \item Rebasing changes the parent so it changes the commit id.
      \item Parent commits are identified by their id.
   \end{itemize}
   {\bf $\rightarrow$ Pushing a rebase commit on a shared branch will break other people workspaces.}
\end{frame}

\begin{frame}
\frametitle{fetch + merge + rebase}
   \mint{bash}|git pull --rebase|
   \begin{itemize}
      \item Same as fetch + merge + rebase.
      \item \alert{Only when you know what you pull} (simple repo, not too many people pushing).
      \item Your local changes (here Start BIG feature) are then applied on top of the remote changes (here Small fix - Tony)
   \end{itemize}
   \begin{minipage}{0.48\linewidth}
      \begin{figure}
         \centering
         \includegraphics[width = \textwidth]{../images/graphs/branches_pull_rebase_before.png}
      \end{figure}
   \end{minipage}
   \begin{minipage}{0.48\linewidth}
      \begin{figure}
         \centering
         \includegraphics[width = \textwidth]{../images/graphs/branches_pull_rebase_after.png}
      \end{figure}
   \end{minipage}
\end{frame}

\begin{frame}
\frametitle{Clean history before pushing}
   \mint{bash}|git rebase -i HEAD~3|
   \begin{itemize}
      \item This is called {\it interactive rebase}.
      \item Can change history on the last n (here 3) commits:
         \begin{itemize}
            \item Squash (merge) commits.
            \item Rewrite commits messages.
            \item Delete commits.
         \end{itemize}
      \item \alert{Prevent other people from seeing your trashy intermediate commits!}
   \end{itemize}
   \begin{minipage}{0.48\linewidth}
      \begin{figure}
         \centering
         \includegraphics[width = \textwidth]{../images/graphs/branches_interactive_before.png}
      \end{figure}
   \end{minipage}
   \begin{minipage}{0.48\linewidth}
      \begin{figure}
         \centering
         \includegraphics[width = \textwidth]{../images/graphs/branches_interactive_after.png}
      \end{figure}
   \end{minipage}
\end{frame}

\begin{frame}
\frametitle{Practice makes perfect}
   https://learngitbranching.js.org/
\end{frame}

%----------------------------------------------------------------------------------------
\subsection{Git workflows}
%----------------------------------------------------------------------------------------

\begin{frame}
\frametitle{Why Git workflows?}
   Git is a toolbox, a team needs to agree on how to use it.
   \begin{itemize}
      \item<1-> How to name the branches?
      \item<2-> Rebase branches or create merge commits?
      \item<3-> When to create tags?
      \item<4-> Each cloud repository come with its own features...
   \end{itemize}
\end{frame}

\begin{frame}
\frametitle{Best practices}
   Not universal but very common.
   \begin{itemize}
      \item Commits messages are meant to be read by others (including the future you).
      \item The 'master' branch should always build/work/succeed.
      \item Implement new features on branches, merge with 'master' when it builds.
      \item Tag new versions/releases of the software.
   \end{itemize}
\end{frame}

\begin{frame}[fragile]
\frametitle{Merge commits or rebase?}
   Branches can be merged by two different commands:
   \begin{columns}
   \begin{column}{0.5\linewidth}
      \mint{bash}|git merge|
      \begin{itemize}
         \item Generates a {\it merge commit} of little value.
         \item The set of commits remains in a separate branch.
      \end{itemize}
      $\rightarrow$ Better for big changes?
   \end{column}
   \begin{column}{0.5\linewidth}
      \mint{bash}|git rebase|
      \begin{itemize}
         \item Git history is lean and direct.
         \item Risk of messing with the 'master' branch history.
      \end{itemize}
      $\rightarrow$ Better for small changes?
   \end{column}
   \end{columns}
\end{frame}

\begin{frame}
\frametitle{Merge Requests / Pull Requests}
   Standard on all cloud repositories (GitHub, GitLab, Bitbucket...).
   \begin{itemize}
      \item Implement changes in a local branch.
      \item Push the branch on the project main repository.
      \item Create a Merge Request on the repository website.
      \item Discuss the proposed changes on the Merge Request webpage.
      \item Maintainer of the project merges the request in the 'master' branch.
   \end{itemize}
\end{frame}

\begin{frame}
\frametitle{Gitflow}
   A complex workflow defined from https://nvie.com/posts/a-successful-git-branching-model/
   \begin{itemize}
      \item Five different kind of branches, with different behavior rules.
      \item Protects some branches for unintended and untested changes.
      \item Ensure reproducibility of versions.
      \item Big mental and organizational overhead.
   \end{itemize}
\end{frame}

\begin{frame}
\frametitle{Gitflow}
   \begin{figure}
      \centering
      \includegraphics[height = 0.9\textheight]{../images/gitflow.png}
   \end{figure}
\end{frame}


\begin{frame}
\frametitle{Prepare for the exercises...}
   Install git in your shell and make sure you can create a repository.

   Create an account on GitHub or GitLab cloud service.
   Ensure you can create a repository and clone it on your computer.
\end{frame}

%----------------------------------------------------------------------------------------
\section{Python}

\end{document}
