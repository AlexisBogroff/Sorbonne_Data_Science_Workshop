\documentclass{beamer}
\usetheme{Warsaw}
\setbeamertemplate{headline}{}

\usepackage{ae,lmodern}
\usepackage[english]{babel}
\usepackage[utf8]{inputenc}
\usepackage[T1]{fontenc}

\usepackage{caption}
\captionsetup[figure]{labelformat=empty}

\PassOptionsToPackage{usenames,dvipsnames}{xcolor}
\usepackage{xcolor,colortbl}
\definecolor{DarkGrey}{HTML}{222222}
\definecolor{DarkBlue}{HTML}{004BA9}
\definecolor{DarkRed}{HTML}{CC1111}
\definecolor{DarkGreen}{HTML}{117711}
\definecolor{DarkOrange}{HTML}{CC7000}
\definecolor{LightGrey}{HTML}{DDDDDD}
\definecolor{LightBlue}{HTML}{F0F8FF}
\definecolor{codegreen}{rgb}{0,0.6,0}
\definecolor{codepurple}{rgb}{0.58,0,0.82}

\usepackage[cache=false]{minted}
\setminted[bash]{
   bgcolor=LightBlue,
   breaklines, breakanywhere,
   frame=single,
   autogobble
}
\usemintedstyle[python]{native}
\setminted[python]{
   bgcolor=black,
   breaklines, breakanywhere,
   autogobble
}

\usepackage{listings}
\usepackage{lstautogobble}
\lstdefinestyle{bash}{
    backgroundcolor=\color{DarkGrey},   
    commentstyle=\color{codegreen},
    keywordstyle=\color{magenta},
    numberstyle=\tiny\color{DarkGrey},
    stringstyle=\color{codepurple},
    basicstyle=\ttfamily\tiny\color{LightGrey},
    escapeinside={\%*}{*)},
    breakatwhitespace=false,         
    breaklines=true,                 
    captionpos=b,                    
    keepspaces=true,                 
    numbers=left,                    
    numbersep=5pt,                  
    showspaces=false,                
    showstringspaces=false,
    showtabs=false,
    showlines=false,
    tabsize=2
}

\usepackage{tikz}
\usetikzlibrary{calc,decorations.pathreplacing,arrows,arrows.meta,shapes,patterns, positioning}
\newcommand\BigLength{14.6em}
\newcommand\Height{2em}
\newcommand\Sep{0.6em}
\newcommand\Center{\BigLength*1/2}
\newcommand\BigBox{\BigLength+\Sep}
\newcommand\HalfBox{\BigLength*1/2-\Sep*1/4}
\newcommand\HalfLength{\BigLength*1/2-\Sep*5/4}
\newcommand\CenterL{\BigLength*1/4-\Sep*1/8}
\newcommand\CenterR{\BigLength*3/4+\Sep*1/8}
\tikzstyle{layer}=[rectangle,thick,text centered,
                     minimum height=\Height,minimum width=\BigLength]
\tikzstyle{short}=[rectangle,thick,text centered,
                     minimum height=\Height,minimum width=\HalfLength]
\tikzstyle{dibox}=[rectangle,thick,semitransparent,
                     minimum height=(\Height+\Sep)*2,minimum width=\BigBox]
\tikzstyle{vmbox}=[rectangle,thick,semitransparent,
                     minimum height=(\Height+\Sep)*3,minimum width=\HalfBox]
\tikzstyle{ctbox}=[rectangle,thick,semitransparent,
                     minimum height=(\Height+\Sep)*2,minimum width=\HalfBox]
\tikzstyle{vebox}=[rectangle,thick,semitransparent,
                     minimum height=(\Height+\Sep)*1,minimum width=\HalfBox]

\usepackage{hyperref}
\usepackage{grffile}


\AtBeginSection[]
{
   \begin{frame}
      \tableofcontents[currentsection]
   \end{frame}
}

\AtBeginSubsection[]
{
   \begin{frame}
      \tableofcontents[currentsection, currentsubsection, sectionstyle=shaded]
   \end{frame}
}

%----------------------------------------------------------------------------------------
\title{Introduction to Data Science}
\subtitle{with Python}
%----------------------------------------------------------------------------------------
\author{Alexis Bogroff}
\date{\today}



\begin{document}

\begin{frame}
   \titlepage
\end{frame}

\begin{frame}
   \tableofcontents
\end{frame}

\section{Linux}
\section{Git}
\section{Python}

% %----------------------------------------------------------------------------------------
% \subsection{Machine Learning: selecting an appropriate model for the problem}
% %----------------------------------------------------------------------------------------

% \begin{frame}[fragile]
%    \frametitle{Type of problem}
%    % Attention je mélange 2 concepts: linéaire et non-linéaire vs regression et classification
%    \begin{itemize}
%       \item Regression: what should be the price of this house? 400k.
%       \item Classification: is this price overvalued? No.
%    \end{itemize}

%    \begin{itemize}
%       \item Linear:
%       \begin{itemize}
%          \item regression: what is the price of this house, based on its surface?
%          \item classification: is this price overvalued, based on the house surface?
%       \end{itemize}
%       \item Non-Linear:
%       \begin{itemize}
%          \item regression: what is the price of this house, based on ?
%          \item classification: is this price overvalued, based on the house surface?
%       \end{itemize}
%    \end{itemize}
% \end{frame}

% \begin{frame}[fragile]
%    \frametitle{Under/Over-fitting}
%    \begin{itemize}
%       \item Complexity:
%       \begin{itemize}
%          \item Type of algorithm
%          \item number of parameters
%       \end{itemize}
%    \end{itemize}
% \end{frame}

% %----------------------------------------------------------------------------------------
% \subsection{Neural Networks}
% %----------------------------------------------------------------------------------------


% \subsubsection{What is a Neural Network?}
% %----------------------------------------

% \begin{frame}[fragile]
%    \frametitle{Perceptron, MLP}
%    \href{https://images.squarespace-cdn.com/content/v1/519a7bc0e4b08ccdf8f31445/1523266910642-HWXCXXOSLG4JLWJ3NPTE/rdn.png?format=1000w}{perceptron}
% \end{frame}

% \begin{frame}[fragile]
%    Is just a linear regression if no use of non-linear functions
%    \frametitle{Non-linear functions}
%    \begin{itemize}
%       \item Sigmoid
%       \item Relu, Tanh
%       \item Softmax
%    \end{itemize}
% \end{frame}

% \begin{frame}[fragile]
%    \frametitle{Deep Neural Network}
% \end{frame}


% %----------------------------------------------------------------------------------------
% \subsection{Pytorch}
% %----------------------------------------------------------------------------------------

% ========================================
% ========================================
% ========================================
% ========================================
% ========================================

%----------------------------------------------------------------------------------------
\subsection{Deep Learning}
%----------------------------------------------------------------------------------------


\begin{frame}[fragile]
   \frametitle{What is a Neural Network in Machine Learning?}
   \begin{itemize}
      \item Also called Multi-layer Perceptron (MLP)
   \end{itemize}
   \begin{center}
      \includegraphics[width=0.8\textwidth]{../images/artificial_neural_networks.jpeg}
   \end{center}
\end{frame}


\begin{frame}[fragile]
   \frametitle{What is a Perceptron?}
   \begin{itemize}
      \item Comparison with a linear regression
      \item Add non linear functions (Sigmoid, Softmax, ReLu, Tanh)
   \end{itemize}
   \begin{center}
      \includegraphics[width=0.6\textwidth]{../images/perceptron.png}
   \end{center}
\end{frame}


\begin{frame}[fragile]
   \frametitle{Fitting (under, good, over)}
   \begin{itemize}
      \item Objective: generalize for production (out-of-sample data)
      \item Simple vs Polynomial regression (few, multiple degrees)
      \item Example house prices (tiny flats, normal, luxury flats)
   \end{itemize}
   \begin{center}
      \includegraphics[width=.8\textwidth]{../images/overfit.png}
   \end{center}
\end{frame}


\begin{frame}[fragile]
   \frametitle{Generalization: Model choice given the task complexity}
   \begin{itemize}
      \item Importance of:
      \begin{itemize}
         \item quantity and initial value of parameters
         \item functions that link the variables and parameters
         \item optimization method
      \end{itemize}
      \item Simpler (often) means more robust
      \item Too simple cannot fit complex tasks
      \item Find the good balance
   \end{itemize}
\end{frame}


\begin{frame}[fragile]
   \frametitle{Adapt model complexity}
   \begin{itemize}
      \item Regularization methods:
      \begin{itemize}
         \item Ridge and Lasso methods for simpler models
         \item Equivalent dropout method for neural networks
      \end{itemize}
      \begin{center}
         \includegraphics[width=.9\textwidth]{../images/ridge.png}
      \end{center}
      \begin{center}
         \includegraphics[width=.9\textwidth]{../images/lasso.png}
      \end{center}
   \end{itemize}
\end{frame}


\begin{frame}[fragile]
   \frametitle{What is Deep Learning?}
   \begin{itemize}
      \item A model (often) using deep nets (>5 hidden layers)
      \item Different levels of representation (raw to elaborate)
      \item See \href{https://www.deeplearningbook.org/}{Y. Goodfellow's book}
   \end{itemize}
   \begin{center}
      \includegraphics[width=.8\textwidth]{../images/deep_net.png}
   \end{center}
\end{frame}


\begin{frame}[fragile]
   \frametitle{Advanced model structures}
   \begin{itemize}
      \item Blocs: dense, convolution, pooling, padding, dropout, etc.
      \begin{center}
         \includegraphics[width=.7\textwidth]{../images/convnet.jpeg}
      \end{center}
      \item VGG16
      \begin{center}
         \includegraphics[width=.9\textwidth]{../images/vgg_16.png}
      \end{center}
   \end{itemize}
\end{frame}



\begin{frame}[fragile]
   \frametitle{Frameworks}
   \begin{itemize}
      \item Tensorflow (Google) and Pytorch (Facebook) frameworks
      \item Faster, simpler and more robust construction of deep learning models
      \item Also help in:
      \begin{itemize}
         \item handling parallel computing, GPU usage for matrix computations
         \item Loading large data sets throughout model training
         \item Visualize and control model fitting / performances
      \end{itemize}
   \end{itemize}
   \begin{center}
      \includegraphics[width=.3\textwidth]{../images/tensorflow_logo.png}
   \end{center}
   \begin{center}
      \includegraphics[width=.25\textwidth]{../images/pytorch_logo_1.png}
   \end{center}
\end{frame}


%----------------------------------------------------------------------------------------
\subsection{Pytorch}
%----------------------------------------------------------------------------------------


\begin{frame}[fragile]
   \frametitle{Building models using Pytorch: Graph}
   \begin{itemize}
      \item Building computational graph upfront, then launch the graph
      \item Dynamic Computational Graphs: built upon running next training iteration\\
      \textit{\href{https://cs230.stanford.edu/section/5/}{See: Graph difference Pytorch and Tensorflow}}
      \begin{center}
         \includegraphics[width=0.6\textwidth]{../images/computational_graph.png}
      \end{center}
   \end{itemize}
\end{frame}


\begin{frame}[fragile]
   \frametitle{Building models using Pytorch: VGG16 Example}
   \begin{itemize}
      \item \textit{\href{https://github.com/pytorch/vision/blob/main/torchvision/models/vgg.py}{Source}}
   \end{itemize}
   \begin{center}
      \includegraphics[width=1\textwidth]{../images/vgg16_pytorch.png}
   \end{center}
\end{frame}

\begin{frame}[fragile]
   \frametitle{\href{https://pytorch.org/docs/stable/nn.html}{Building models using Pytorch: Components}}
   \begin{itemize}
      \item Layers
      \begin{itemize}
         \item Linear
         \begin{itemize}
            \item \href{https://pytorch.org/docs/stable/generated/torch.nn.Linear.html}{Linear} (dense layer vs sparse layer)
         \end{itemize}

         \item Convolutions
         \begin{itemize}
               \item \href{https://pytorch.org/docs/stable/generated/torch.nn.Conv2d.html#torch.nn.Conv2d}{Conv2d}
               \item \href{https://pytorch.org/docs/stable/generated/torch.nn.AvgPool2d.html#torch.nn.AvgPool2d}{AvgPool2d}
               \item \href{https://pytorch.org/docs/stable/generated/torch.nn.MaxPool2d.html#torch.nn.MaxPool2d}{MaxPool2d}
         \end{itemize}

         \item Recurrent
         \begin{itemize}
            \item \href{https://pytorch.org/docs/stable/generated/torch.nn.RNN.html#torch.nn.RNN}{RNN}
            \item \href{https://pytorch.org/docs/stable/generated/torch.nn.LSTM.html#torch.nn.LSTM}{LSTM}
         \end{itemize}

         \item Transformations
         \begin{itemize}
            \item \href{https://pytorch.org/docs/stable/generated/torch.nn.BatchNorm2d.html#torch.nn.BatchNorm2d}{BatchNorm2d}
            \item \href{https://pytorch.org/docs/stable/generated/torch.nn.Dropout2d.html#torch.nn.Dropout2d}{Dropout2d}
            \item \href{https://pytorch.org/docs/stable/generated/torch.flatten.html}{Flatten}
         \end{itemize}
      \end{itemize}

      \item (Non-linear) Activation functions:
      \begin{itemize}
         \item \href{https://pytorch.org/docs/stable/generated/torch.nn.Sigmoid.html}{Sigmoid}
         \item \href{https://pytorch.org/docs/stable/generated/torch.nn.ReLU.html}{ReLu}
         \item \href{https://pytorch.org/docs/stable/generated/torch.nn.Tanh.html#torch.nn.Tanh}{Tanh}
      \end{itemize}

      \item \href{https://pytorch.org/docs/stable/nn.html#loss-functions}{Loss functions}: MSELoss, CrossEntropyLoss
   \end{itemize}
\end{frame}

\begin{frame}[fragile]
   \frametitle{Building models using Pytorch: Optimization loop}
   \begin{center}
      \includegraphics[width=1\textwidth]{../images/optimization_loop_pytorch.png}
   \end{center}
\end{frame}


\begin{frame}[fragile]
   \frametitle{\href{https://pytorch.org/tutorials/beginner/basics/intro.html}{Pytorch Inner Workings}}
   \begin{itemize}
      \item \href{https://pytorch.org/tutorials/beginner/basics/tensorqs_tutorial.html}{Tensors: like Numpy Arrays + enable computational graphs and autograd}\\
      \includegraphics[width=.5\textwidth]{../images/tensor_pytorch.png}
      \item Automatic Differenciation (AutoGrad)
      \includegraphics[width=.8\textwidth]{../images/define_parameters.png}
      \includegraphics[width=.35\textwidth]{../images/get_gradients.png}
   \end{itemize}
\end{frame}

\begin{frame}[fragile]
   \frametitle{\href{https://pytorch.org/docs/stable/optim.html}{Optimizers}}
   \begin{itemize}
         \item SGD
         \item Adam
   \end{itemize}
   \includegraphics[width=.9\textwidth]{../images/optimizers_pytorch.pdf}
   \includegraphics[width=.6\textwidth]{../images/optimization.jpeg}
\end{frame}


\begin{frame}[fragile]
   \frametitle{\href{https://pytorch.org/tutorials/beginner/basics/data_tutorial.html}{Datasets and DataLoaders}}
   \begin{itemize}
      \item Batches\\
      \includegraphics[width=.8\textwidth]{../images/dataloaders.png}
      \item \href{https://pytorch.org/tutorials/beginner/basics/data_tutorial.html}{Specific transformations}
      \includegraphics[width=.9\textwidth]{../images/set_dataloaders.png}
   \end{itemize}
\end{frame}


\begin{frame}[fragile]
   \frametitle{Working with a trained model}
   \begin{itemize}
      \item Save, Load and Use Model
      \item Transfer Learning (freezing weights)
   \end{itemize}
\end{frame}



\begin{frame}[fragile]
   \frametitle{Examples}
   \begin{itemize}
      \item \href{https://github.com/pytorch/examples/blob/main/mnist/main.py}{Example applied to MNIST - Github}
      \item \href{https://github.com/pytorch/examples/blob/main/mnist/main.py}{Example applied to MNIST - Google Colab}
   \end{itemize}
\end{frame}

\begin{frame}[fragile]
   \frametitle{Further}
   \begin{itemize}
      \item \href{https://pytorch.org/tutorials/}{Official documentation}
   \end{itemize}
\end{frame}


%----------------------------------------------------------------------------------------
\section{Conclusion}
%----------------------------------------------------------------------------------------


\begin{frame}
\frametitle{Quick Summary - Linux}
   Linux OS:
   \begin{itemize}
      \item Open source development and production platform.
      \item Designed by IT for IT.
   \end{itemize}
   Linux shell:
   \begin{itemize}
      \item Precise way to manipulate basic IT tools.
      \item Can be reproduced and automatized.
   \end{itemize}
\end{frame}

\begin{frame}
\frametitle{Quick Summary - Git}
   Decentralized versioning system:
   \begin{itemize}
      \item Follow your own changes.
      \item Share your work with others.
      \item Can experiment in branches.
   \end{itemize}
\end{frame}

\begin{frame}
\frametitle{Quick Summary - Python}
   High-level language:
   \begin{itemize}
      \item Can do everything from script to web server.
      \item Define virtual environment with multiple libraries.
      \item Can use imperative, object or functional programming.
   \end{itemize}
\end{frame}

\begin{frame}
\frametitle{Sample use cases}
   Web server
   \begin{itemize}
      \item Implement it in Python using the Django framework.
      \item Share it on a private Github, using merge requests for changes.
      \item Install, start and maintain it using shell commands.
   \end{itemize}
\end{frame}

\begin{frame}
\frametitle{Sample use cases}
   Data analysis
   \begin{itemize}
      \item Download data using linux shell.
      \item Parse and compute data in Python, using a Jupyter notebook.
      \item Share it on Gitlab.
   \end{itemize}
\end{frame}

\begin{frame}
\frametitle{Sample use cases}
   Data monitoring prototype
   \begin{itemize}
      \item Implement data loading and parsing in a Python script.
      \item Execute every night by a Linux shell command.
      \item Deliver it in the company private Gitlab.
   \end{itemize}
\end{frame}

\begin{frame}
\frametitle{Questions?}
   Any question?
\end{frame}

\begin{frame}
\frametitle{Reminder: final project}
   Its text will be published soon. \\
   \\
   You can start working on it as soon as the text is published.
\end{frame}


\end{document}